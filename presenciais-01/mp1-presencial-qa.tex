
\documentclass[
	% -- opções da classe memoir --
    article,            % artigo academico
	12pt,				% tamanho da fonte
	%openright,			% capítulos começam em pág ímpar (insere página vazia caso preciso)
	oneside,			% para impressão em recto e verso. Oposto a oneside (twoside)
	a4paper,			% tamanho do papel. 
	% -- opções da classe abntex2 --
	chapter=TITLE,		% títulos de capítulos convertidos em letras maiúsculas
	section=TITLE,		% títulos de seções convertidos em letras maiúsculas
	%subsection=TITLE,	% títulos de subseções convertidos em letras maiúsculas
	%subsubsection=TITLE,% títulos de subsubseções convertidos em letras maiúsculas
	% -- opções do pacote babel --
	english,			% idioma adicional para hifenização
	french,				% idioma adicional para hifenização
	spanish,			% idioma adicional para hifenização
	brazil				% o último idioma é o principal do documento
	]{abntex2}

% ---
% Pacotes básicos 
% ---
\usepackage{times}				% Usa a fonte Times Roman			
\usepackage[T1]{fontenc}		% Selecao de codigos de fonte.
\usepackage[utf8]{inputenc}		% Codificacao do documento (conversão automática dos acentos)
\usepackage{indentfirst}		% Indenta o primeiro parágrafo de cada seção.
\usepackage{color}				% Controle das cores
\usepackage{graphicx}			% Inclusão de gráficos
\usepackage{microtype} 			% para melhorias de justificação
% ---

% ---
% Pacotes de citações
% ---
\usepackage[brazilian,hyperpageref]{backref}	 % Paginas com as citações na bibl
\usepackage[alf]{abntex2cite}	% Citações padrão ABNT

% --- 
% CONFIGURAÇÕES DE PACOTES
% --- 

% ---
% Configurações do pacote backref
% Usado sem a opção hyperpageref de backref
\renewcommand{\backrefpagesname}{Citado na(s) página(s):~}
% Texto padrão antes do número das páginas
\renewcommand{\backref}{}
% Define os textos da citação
\renewcommand*{\backrefalt}[4]{
	\ifcase #1 %
		Nenhuma citação no texto.%
	\or
		Citado na página #2.%
	\else
		Citado #1 vezes nas páginas #2.%
	\fi}%
% ---
% ---
% FORMATAÇAO FLAM
% ---
\setlength{\parindent}{1.25cm}
\setlength{\parskip}{0.5cm}
\setlength\afterchapskip{\lineskip}
\setlrmarginsandblock{3cm}{2cm}{*}
\setulmarginsandblock{3cm}{2cm}{*}
\checkandfixthelayout
\renewcommand{\ABNTEXchapterfont}{\normalfont}
\renewcommand{\ABNTEXsectionfontsize}{\large\bfseries}
\renewcommand{\cftsectionfont}{\bfseries\MakeTextUppercase}
\renewcommand{\ABNTEXsubsectionfontsize}{\normalsize}
\renewcommand{\cftsubsectionfont}{\normalfont\MakeTextUppercase} % Tirar negrito das subsecoes no sumario
\renewcommand{\ABNTEXsubsubsectionfontsize}{\normalsize\bfseries}
\renewcommand{\cftsubsubsectionfont}{\bfseries} % Tirar negrito das subsecoes no sumario
% ---
% Informações de dados para CAPA e FOLHA DE ROSTO
% ---
\titulo{METODOLOGIA DA PESQUISA \\ QUESTÃO DISSERTATIVA}
\autor{GABRIEL CARDOSO DOS SANTOS FALEIRO}
\local{ARUJÁ-SP}
\data{2025}
\instituicao{%
  FLAM - FACULDADE LATINO AMERICANA
}
\tipotrabalho{QUESTÃO DISSERTATIVA}
% O preambulo deve conter o tipo do trabalho, o objetivo, 
% o nome da instituição e a área de concentração 
\preambulo{Trabalho da disciplina de Metodologia da Pesquisa, solicitado pela prof. Dra. Inês Murad.}
% ---


% ---
% Configurações de aparência do PDF final

% alterando o aspecto da cor azul
\definecolor{blue}{RGB}{41,5,195}

% informações do PDF
\makeatletter
\hypersetup{
     	%pagebackref=true,
		pdftitle={\@title}, 
		pdfauthor={\@author},
    	pdfsubject={\imprimirpreambulo},
	    pdfcreator={Gabriel Cardoso dos Santos Faleiro},
		pdfkeywords={abnt}{latex}{abntex}{abntex2}{trabalho acadêmico}, 
		colorlinks=true,       		% false: boxed links; true: colored links
    	linkcolor=blue,          	% color of internal links
    	citecolor=blue,        		% color of links to bibliography
    	filecolor=magenta,      		% color of file links
		urlcolor=blue,
		bookmarksdepth=4
}
\makeatother
% ---
% ---
% compila o indice
% ---
\makeindex
% ---

% ----
% Início do documento
% ----
\begin{document}

\citeoption{abnt-full-initials=yes}


% Seleciona o idioma do documento (conforme pacotes do babel)
%\selectlanguage{english}
\selectlanguage{brazil}
% ----------------------------------------------------------
% ELEMENTOS PRÉ-TEXTUAIS
% ----------------------------------------------------------
% \pretextual

% ---
% Capa
% ---
\imprimircapa
% ---

% ---
% Folha de rosto
% (o * indica que haverá a ficha bibliográfica)
% ---
% \imprimirfolhaderosto*
\imprimirfolhaderosto
% ---

% ---
% inserir o sumario
% ---
% \pdfbookmark[0]{\contentsname}{toc}
% \tableofcontents*
% \cleardoublepage
% ---

% ----------------------------------------------------------
% ELEMENTOS TEXTUAIS
% ----------------------------------------------------------
\textual
\pagestyle{simple}

% ----------------------------------------------------------
% Introdução (mas presente no Sumário)

\section*{AS DUAS NATUREZAS HUMANAS}
% [...] devemos ter em conta que toda pessoa cristã possui duas naturezas: uma espiritual e outra corporal. Tendo em vista a alma, ela é designada de ser humano espiritual, novo e interior; segundo a carne e o sangue, é chamada de ser humano corporal, velho e exterior. [...] Neste ponto fica claro que nenhuma coisa externa, seja qual for, pode fazer dele alguém agradável a Deus ou livre; pois nem sua piedade e liberdade, nem sua maldade e cativeiro são corpóreos ou externos. Que proveito tem a alma se o corpo é livre, forte e saudável, se come, bebe e vive como quer? Inversamente, que dano sofre a alma se o corpo, contra sua vontade, está aprisionado, enfermo e fraco, padecendo fome, sede e sofrimentos? Nada disso atinge a alma de maneira alguma, seja para libertá-la ou escravizá-la, seja para torná-la agradável a Deus ou má (LUTERO, 1998. p. 8).

% LUTERO, Martinho; PASCHOAL, Erlon José. Da liberdade do cristão: prefácio a Bíblia. SciELO-Editora UNESP, 1998.

Segundo \citeonline{LUTERO}, cristãos são dotados de duas naturezas que se diferenciam em uma ser espiritual e outra carnal. Também propõe que, dada essa natureza distinta e separada, fatores externos relacionados a materialidade não podem fazer de nenhum cristão agradável aos olhos de Deus. Para esta conclusão, Lutero se utiliza das proposições lógicas:
\begin{citacao}
Que proveito tem a alma se o corpo é livre, forte e saudável, se come, bebe e vive como quer? Inversamente, que dano sofre a alma se o corpo, contra sua vontade, está aprisionado, enfermo e fraco, padecendo fome, sede e sofrimentos? Nada disso atinge a alma de maneira alguma, seja para libertá-la ou escravizá-la, seja para torná-la agradável a Deus ou má \cite{LUTERO}
\end{citacao}
% ----------------------------------------------------------

\pagebreak
\renewcommand{\bibname}{{REFER\^ENCIAS}}
\bibliography{template_flam.bib}

\end{document}
