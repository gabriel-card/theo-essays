
\documentclass[
	% -- opções da classe memoir --
    article,            % artigo academico
	12pt,				% tamanho da fonte
	%openright,			% capítulos começam em pág ímpar (insere página vazia caso preciso)
	oneside,			% para impressão em recto e verso. Oposto a oneside (twoside)
	a4paper,			% tamanho do papel. 
	% -- opções da classe abntex2 --
	chapter=TITLE,		% títulos de capítulos convertidos em letras maiúsculas
	section=TITLE,		% títulos de seções convertidos em letras maiúsculas
	%subsection=TITLE,	% títulos de subseções convertidos em letras maiúsculas
	%subsubsection=TITLE,% títulos de subsubseções convertidos em letras maiúsculas
	% -- opções do pacote babel --
	english,			% idioma adicional para hifenização
	french,				% idioma adicional para hifenização
	spanish,			% idioma adicional para hifenização
	brazil				% o último idioma é o principal do documento
	]{abntex2}

% ---
% Pacotes básicos 
% ---
\usepackage{times}				% Usa a fonte Times Roman			
\usepackage[T1]{fontenc}		% Selecao de codigos de fonte.
\usepackage[utf8]{inputenc}		% Codificacao do documento (conversão automática dos acentos)
\usepackage{indentfirst}		% Indenta o primeiro parágrafo de cada seção.
\usepackage{color}				% Controle das cores
\usepackage{graphicx}			% Inclusão de gráficos
\usepackage{microtype} 			% para melhorias de justificação
% ---

% ---
% Pacotes de citações
% ---
\usepackage[brazilian,hyperpageref]{backref}	 % Paginas com as citações na bibl
\usepackage[alf]{abntex2cite}	% Citações padrão ABNT

% --- 
% CONFIGURAÇÕES DE PACOTES
% --- 

% ---
% Configurações do pacote backref
% Usado sem a opção hyperpageref de backref
\renewcommand{\backrefpagesname}{Citado na(s) página(s):~}
% Texto padrão antes do número das páginas
\renewcommand{\backref}{}
% Define os textos da citação
\renewcommand*{\backrefalt}[4]{
	\ifcase #1 %
		Nenhuma citação no texto.%
	\or
		Citado na página #2.%
	\else
		Citado #1 vezes nas páginas #2.%
	\fi}%
% ---
% ---
% FORMATAÇAO FLAM
% ---
\setlength{\parindent}{1.25cm}
\setlength{\parskip}{0.5cm}
\setlength\afterchapskip{\lineskip}
\setlrmarginsandblock{3cm}{2cm}{*}
\setulmarginsandblock{3cm}{2cm}{*}
\checkandfixthelayout
\renewcommand{\ABNTEXchapterfont}{\normalfont}
\renewcommand{\ABNTEXsectionfontsize}{\large\bfseries}
\renewcommand{\cftsectionfont}{\bfseries\MakeTextUppercase}
\renewcommand{\ABNTEXsubsectionfontsize}{\normalsize}
\renewcommand{\cftsubsectionfont}{\normalfont\MakeTextUppercase} % Tirar negrito das subsecoes no sumario
\renewcommand{\ABNTEXsubsubsectionfontsize}{\normalsize\bfseries}
\renewcommand{\cftsubsubsectionfont}{\bfseries} % Tirar negrito das subsecoes no sumario
% ---
% Informações de dados para CAPA e FOLHA DE ROSTO
% ---
\titulo{ANTIGO TESTAMENTO 1: PENTATEUCO E HISTÓRICOS \\ QUESTÃO DISSERTATIVA}
\autor{GABRIEL CARDOSO DOS SANTOS FALEIRO}
\local{ARUJÁ-SP}
\data{2025}
\instituicao{%
  FLAM - FACULDADE LATINO AMERICANA
}
\tipotrabalho{QUESTÃO DISSERTATIVA}
% O preambulo deve conter o tipo do trabalho, o objetivo, 
% o nome da instituição e a área de concentração 
\preambulo{Trabalho da disciplina de Antigo Testamento 1: Pentateuco e Históricos, solicitado pelo prof. Dr. Fábio Ito.}
% ---


% ---
% Configurações de aparência do PDF final

% alterando o aspecto da cor azul
\definecolor{blue}{RGB}{41,5,195}

% informações do PDF
\makeatletter
\hypersetup{
     	%pagebackref=true,
		pdftitle={\@title}, 
		pdfauthor={\@author},
    	pdfsubject={\imprimirpreambulo},
	    pdfcreator={Gabriel Cardoso dos Santos Faleiro},
		pdfkeywords={abnt}{latex}{abntex}{abntex2}{trabalho acadêmico}, 
		colorlinks=true,       		% false: boxed links; true: colored links
    	linkcolor=blue,          	% color of internal links
    	citecolor=blue,        		% color of links to bibliography
    	filecolor=magenta,      		% color of file links
		urlcolor=blue,
		bookmarksdepth=4
}
\makeatother
% ---
% ---
% compila o indice
% ---
\makeindex
% ---

% ----
% Início do documento
% ----
\begin{document}

\citeoption{abnt-full-initials=yes}


% Seleciona o idioma do documento (conforme pacotes do babel)
%\selectlanguage{english}
\selectlanguage{brazil}
% ----------------------------------------------------------
% ELEMENTOS PRÉ-TEXTUAIS
% ----------------------------------------------------------
% \pretextual

% ---
% Capa
% ---
\imprimircapa
% ---

% ---
% Folha de rosto
% (o * indica que haverá a ficha bibliográfica)
% ---
% \imprimirfolhaderosto*
\imprimirfolhaderosto
% ---

% ---
% inserir o sumario
% ---
% \pdfbookmark[0]{\contentsname}{toc}
% \tableofcontents*
% \cleardoublepage
% ---

% ----------------------------------------------------------
% ELEMENTOS TEXTUAIS
% ----------------------------------------------------------
\textual
\pagestyle{simple}

% ----------------------------------------------------------
% Introdução (mas presente no Sumário)

\section*{A VIDA COMO ATO CONTÍNUO DE DEUS}
% Uma teologia da criação deve considerar a criação da vida no livro de Gênesis. O homem somente recebeu vida porque Deus lhe deu, soprando em suas narinas o fôlego de vida (Gn 2.7). O redator de Jó amplia tal compreensão com a fala de Eliú quando afirma que a vida é um dom constante de Deus: “O Espírito de Deus me fez, e o sopro do Todo Poderoso me dá vida” (Jó 33.4). Exponha, em 20 linhas, de que modo o dom da vida é um ato contínuo de Deus e não se refere somente a um momento fundante do mundo.

Entender o dom da vida como uma dádiva divina que aconteceu em apenas um único momento na história é entender que o Criador não interage mais com sua criação. A idéia deísta de não existir nenhum tipo de intervenção divina desde o momento da criação é fundamentalmente contrária a narrativa da redenção que encontramos na Bíblia.

Encontramos na narrativa bíblica, em uma leitura panorâmica, a queda da humanidade ao pecado e um plano redentor de Deus para resgatar e restaurar sua criação, que culmina na vinda, morte, ressurreição e ascensão de Jesus Cristo. Ora, se existe um plano vindo diretamente do Criador para restaurar sua própria criação, já se torna impossível a compreensão deísta, afinal, um deus que não intervém em sua criação não estaria preocupado em restaurá-la. Portanto, ao mergulharmos nos textos do antigo e do novo testamento, temos testemunhas da continuidade dessas intervenções que guiam a humanidade a sua salvação de seu estado de queda. Em outras palavras, a restauração da vida que, antes de tudo, Deus nos deu.

% "Já estou crucificado com Cristo; e vivo, não mais eu, mas Cristo vive em mim; e a vida que agora vivo na carne, vivo-a pela fé do Filho de Deus" (Gl 2, 20)\footnote{GÁLATAS. In: A BÍBLIA SAGRADA: Almeida Corrigida Fiel. São Paulo, 2011.}
Paulo, em sua carta aos gálatas, descreve de forma palpável como esse ato contínuo de Deus é presente no momento que vivemos: "Já estou crucificado com Cristo; e vivo, não mais eu, mas Cristo vive em mim; e a vida que agora vivo na carne, vivo-a pela fé do Filho de Deus" (Gl 2, 20)\footnote{GÁLATAS. In: A BÍBLIA SAGRADA: Almeida Corrigida Fiel. São Paulo, 2011.}. É através da fé em Cristo que sua vida passada é considerada morta e a única coisa viva dentro dele é a vida de Cristo. Ora, se quem vive em mim hoje é o próprio Cristo, o Verbo que por Ele tudo foi criado, a dádiva dessa vida é constante e eterna e não apenas um momento único no início de todas as coisas.


% ----------------------------------------------------------

\pagebreak
\renewcommand{\bibname}{{REFER\^ENCIAS}}
\bibliography{template_flam.bib}

\end{document}
