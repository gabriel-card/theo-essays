
\documentclass[
	% -- opções da classe memoir --
    article,            % artigo academico
	12pt,				% tamanho da fonte
	%openright,			% capítulos começam em pág ímpar (insere página vazia caso preciso)
	oneside,			% para impressão em recto e verso. Oposto a oneside (twoside)
	a4paper,			% tamanho do papel. 
	% -- opções da classe abntex2 --
	%chapter=TITLE,		% títulos de capítulos convertidos em letras maiúsculas
	%section=TITLE,		% títulos de seções convertidos em letras maiúsculas
	%subsection=TITLE,	% títulos de subseções convertidos em letras maiúsculas
	%subsubsection=TITLE,% títulos de subsubseções convertidos em letras maiúsculas
	% -- opções do pacote babel --
	english,			% idioma adicional para hifenização
	french,				% idioma adicional para hifenização
	spanish,			% idioma adicional para hifenização
	brazil				% o último idioma é o principal do documento
	]{abntex2}

% ---
% Pacotes básicos 
% ---
\usepackage{lmodern}			% Usa a fonte Latin Modern			
\usepackage[T1]{fontenc}		% Selecao de codigos de fonte.
\usepackage[utf8]{inputenc}		% Codificacao do documento (conversão automática dos acentos)
\usepackage{indentfirst}		% Indenta o primeiro parágrafo de cada seção.
\usepackage{color}				% Controle das cores
\usepackage{graphicx}			% Inclusão de gráficos
\usepackage{microtype} 			% para melhorias de justificação
% ---

% ---
% Pacotes de citações
% ---
\usepackage[brazilian,hyperpageref]{backref}	 % Paginas com as citações na bibl
\usepackage[alf]{abntex2cite}	% Citações padrão ABNT

% --- 
% CONFIGURAÇÕES DE PACOTES
% --- 

% ---
% Configurações do pacote backref
% Usado sem a opção hyperpageref de backref
\renewcommand{\backrefpagesname}{Citado na(s) página(s):~}
% Texto padrão antes do número das páginas
\renewcommand{\backref}{}
% Define os textos da citação
\renewcommand*{\backrefalt}[4]{
	\ifcase #1 %
		Nenhuma citação no texto.%
	\or
		Citado na página #2.%
	\else
		Citado #1 vezes nas páginas #2.%
	\fi}%
% ---

% ---
% Informações de dados para CAPA e FOLHA DE ROSTO
% ---
\titulo{METODOLOGIA DA PESQUISA \\ QUESTÃO ABERTA 01}
\autor{GABRIEL CARDOSO DOS SANTOS FALEIRO}
\local{ARUJÁ-SP}
\data{2024}
\instituicao{%
  FLAM - FACULDADE LATINO AMERICANA
}
\tipotrabalho{QUESTÃO ABERTA 01}
% O preambulo deve conter o tipo do trabalho, o objetivo, 
% o nome da instituição e a área de concentração 
\preambulo{Trabalho da disciplina de Metodologia da Pesquisa, solicitado pela prof. Inês Murad.}
% ---


% ---
% Configurações de aparência do PDF final

% alterando o aspecto da cor azul
\definecolor{blue}{RGB}{41,5,195}

% informações do PDF
\makeatletter
\hypersetup{
     	%pagebackref=true,
		pdftitle={\@title}, 
		pdfauthor={\@author},
    	pdfsubject={\imprimirpreambulo},
	    pdfcreator={LaTeX with abnTeX2},
		pdfkeywords={abnt}{latex}{abntex}{abntex2}{trabalho acadêmico}, 
		colorlinks=true,       		% false: boxed links; true: colored links
    	linkcolor=blue,          	% color of internal links
    	citecolor=blue,        		% color of links to bibliography
    	filecolor=magenta,      		% color of file links
		urlcolor=blue,
		bookmarksdepth=4
}
\makeatother
% ---
% ---
% compila o indice
% ---
\makeindex
% ---

% ----
% Início do documento
% ----
\begin{document}

\citeoption{abnt-full-initials=yes}


% Seleciona o idioma do documento (conforme pacotes do babel)
%\selectlanguage{english}
\selectlanguage{brazil}
% ----------------------------------------------------------
% ELEMENTOS PRÉ-TEXTUAIS
% ----------------------------------------------------------
% \pretextual

% ---
% Capa
% ---
\imprimircapa
% ---

% ---
% Folha de rosto
% (o * indica que haverá a ficha bibliográfica)
% ---
% \imprimirfolhaderosto*
\imprimirfolhaderosto
% ---

% ---
% inserir o sumario
---
\pdfbookmark[0]{\contentsname}{toc}
\tableofcontents*
\cleardoublepage
% ---

% ----------------------------------------------------------
% ELEMENTOS TEXTUAIS
% ----------------------------------------------------------
\textual

% ----------------------------------------------------------
% Introdução (mas presente no Sumário)
% ----------------------------------------------------------
\section{TEMA}
O tema será: "A formação de comunidades cristãs sob uma perspectiva escatológica: consequências da expectativa escatológica nas origens de uma igreja.".

\section{JUSTIFICATIVA}
Será analisado neste Trabalho de Conclusão de Curso como se dá a formação de comunidades cristãs que, em sua incepção, possuem grande carga de expectativa escatológica. Será explorado primariamente a formação da Igreja Primitiva, não só como a primeira comunidade cristã mas também por sua grande força teológica na expectativa da segunda vinda de Cristo; e em seguida será pesquisado outras comunidades que tiveram, também, como principal característica ou principal motivador de formação comunitária esta mesma expectativa.

Para elaborar o trabalho, será utilizado os método de pesquisa documental e descritiva que buscará detalhar e montar qual é o cenário propício para a formação de uma comunidade onde seus indivíduos, em conjunto, aguardam a volta do Messias baseado no estudo de comunidades cristãs que tiveram esta mesma crença como principal. O empenho estará em entender não só as circunstâncias prévias daqueles indivíduos que se juntaram em uma igreja ou comunidade mas também em como essa comunidade existia em si mesma e dentro da sociedade que estavam inseridos.

Após entender como se dá este cenário propício para a formação destas igrejas, será também elaborado uma diferenciação entre essas comunidades com as demais comunidades que não possuem esta mesma crença como basilar (apesar de também terem a expectativa); características como estrutura eclesiástica, impacto dessas comunidades na sociedade e o dia-a-dia de seus membros.

Por fim, espera-se contribuir não apenas com uma melhor compreensão acerca destas comunidades mas também demonstrar como nós, pertencentes a instituições cristãs que se empenham para se permanecer perene, podemos repensar em como adotar novas práticas comunitárias e estruturas eclesiásticas ou transformar as existentes para que consigamos retomar a expectativa da volta de Cristo como algo caro e importante à nossa fé.

\pagebreak

\end{document}
