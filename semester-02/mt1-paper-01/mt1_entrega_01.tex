
\documentclass[
	% -- opções da classe memoir --
    article,            % artigo academico
	12pt,				% tamanho da fonte
	%openright,			% capítulos começam em pág ímpar (insere página vazia caso preciso)
	oneside,			% para impressão em recto e verso. Oposto a oneside (twoside)
	a4paper,			% tamanho do papel. 
	% -- opções da classe abntex2 --
	chapter=TITLE,		% títulos de capítulos convertidos em letras maiúsculas
	section=TITLE,		% títulos de seções convertidos em letras maiúsculas
	%subsection=TITLE,	% títulos de subseções convertidos em letras maiúsculas
	%subsubsection=TITLE,% títulos de subsubseções convertidos em letras maiúsculas
	% -- opções do pacote babel --
	english,			% idioma adicional para hifenização
	french,				% idioma adicional para hifenização
	spanish,			% idioma adicional para hifenização
	brazil				% o último idioma é o principal do documento
	]{abntex2}

% ---
% Pacotes básicos 
% ---
\usepackage{times}				% Usa a fonte Times Roman			
\usepackage[T1]{fontenc}		% Selecao de codigos de fonte.
\usepackage[utf8]{inputenc}		% Codificacao do documento (conversão automática dos acentos)
\usepackage{indentfirst}		% Indenta o primeiro parágrafo de cada seção.
\usepackage{color}				% Controle das cores
\usepackage{graphicx}			% Inclusão de gráficos
\usepackage{microtype} 			% para melhorias de justificação
% ---

% ---
% Pacotes de citações
% ---
\usepackage[brazilian,hyperpageref]{backref}	 % Paginas com as citações na bibl
\usepackage[alf]{abntex2cite}	% Citações padrão ABNT

% --- 
% CONFIGURAÇÕES DE PACOTES
% --- 

% ---
% Configurações do pacote backref
% Usado sem a opção hyperpageref de backref
\renewcommand{\backrefpagesname}{Citado na(s) página(s):~}
% Texto padrão antes do número das páginas
\renewcommand{\backref}{}
% Define os textos da citação
\renewcommand*{\backrefalt}[4]{}%
% ---
% ---
% FORMATAÇAO FLAM
% ---
\setlength{\parindent}{1.25cm}
\setlength{\parskip}{0.5cm}
\setlength\afterchapskip{\lineskip}
\setlrmarginsandblock{3cm}{2cm}{*}
\setulmarginsandblock{3cm}{2cm}{*}
\checkandfixthelayout
\renewcommand{\ABNTEXchapterfont}{\normalfont}
\renewcommand{\ABNTEXsectionfontsize}{\large\bfseries}
\renewcommand{\cftsectionfont}{\bfseries\MakeTextUppercase}
\renewcommand{\ABNTEXsubsectionfontsize}{\normalsize}
\renewcommand{\cftsubsectionfont}{\normalfont\MakeTextUppercase} % Tirar negrito das subsecoes no sumario
\renewcommand{\ABNTEXsubsubsectionfontsize}{\normalsize\bfseries}
\renewcommand{\cftsubsubsectionfont}{\bfseries} % Tirar negrito das subsecoes no sumario
% ---
% Informações de dados para CAPA e FOLHA DE ROSTO
% ---
\titulo{MISSÕES TRANSCULTURAIS \\ ENTREGA 1}
\autor{GABRIEL CARDOSO DOS SANTOS FALEIRO}
\local{ARUJÁ-SP}
\data{2025}
\instituicao{%
  FLAM - FACULDADE LATINO AMERICANA
}
\tipotrabalho{ENTREGA 1}
% O preambulo deve conter o tipo do trabalho, o objetivo, 
% o nome da instituição e a área de concentração 
\preambulo{Trabalho da disciplina de Missões Transculturais, solicitado pelo prof. Dr. Edilson Botelho Nogueira.}
% ---


% ---
% Configurações de aparência do PDF final

% alterando o aspecto da cor azul
\definecolor{blue}{RGB}{41,5,195}

% informações do PDF
\makeatletter
\hypersetup{
     	%pagebackref=true,
		pdftitle={\@title}, 
		pdfauthor={\@author},
    	pdfsubject={\imprimirpreambulo},
	    pdfcreator={Gabriel Cardoso dos Santos Faleiro},
		pdfkeywords={abnt}{latex}{abntex}{abntex2}{trabalho acadêmico}, 
		colorlinks=true,       		% false: boxed links; true: colored links
    	linkcolor=blue,          	% color of internal links
    	citecolor=blue,        		% color of links to bibliography
    	filecolor=magenta,      		% color of file links
		urlcolor=blue,
		bookmarksdepth=4
}
\makeatother
% ---
% ---
% compila o indice
% ---
\makeindex
% ---

% ----
% Início do documento
% ----
\begin{document}

\citeoption{abnt-full-initials=yes}


% Seleciona o idioma do documento (conforme pacotes do babel)
%\selectlanguage{english}
\selectlanguage{brazil}
% ----------------------------------------------------------
% ELEMENTOS PRÉ-TEXTUAIS
% ----------------------------------------------------------
% \pretextual

% ---
% Capa
% ---
\imprimircapa
% ---

% ---
% Folha de rosto
% (o * indica que haverá a ficha bibliográfica)
% ---
% \imprimirfolhaderosto*
\imprimirfolhaderosto
% ---

% ---
% inserir o sumario
% ---
\pdfbookmark[0]{\contentsname}{toc}
\tableofcontents*
\cleardoublepage
% ---

% ----------------------------------------------------------
% ELEMENTOS TEXTUAIS
% ----------------------------------------------------------
\textual
\pagestyle{simple}

% ----------------------------------------------------------
% Introdução (mas presente no Sumário)

% INTRODUÇÃO
% ● Apresentação do Tema: O que é a Missio Dei?
% 1. ISRAEL COMO CENTRO DA MISSÃO NO ANTIGO TESTAMENTO
% 1.1. ISRAEL: ELEITO PARA TESTEMUNHAR
% ● A Eleição de Israel como Reino de Sacerdotes, Êxodo 19:5–6
% ● A vocação missionária de Israel, Salmo 67
% ● Jerusalém como centro redentivo para o mundo Salmo 87
% 1. 2. DEUS EM MISSÃO BUSCA TODAS AS NAÇÕES DO MUNDO
% ● A mensagem dos Profetas aos outros povos da terra.
% ● O Cativeiro de Israel como oportunidade missional
% ● O Silencio missional de Deus e seu preparo para a Plenitude do Tempo

\section{INTRODUÇÃO}
\emph{Missio Dei}\footnote{Em tradução livre: Missão de Deus.}, termo criado pelo teólogo alemão Karl Hartenstein, carrega consigo uma mudança significativa que acontecia em sua época na perspectiva missiológica da cristandade. Hartenstein argumenta que a missão da Igreja não parte da humanidade, mas sim de uma ação direta de Deus na história. Essa ação contínua está presente em toda a narrativa bíblica, passando pela escolha de Israel como povo de Deus para que testemunhassem sobre Ele para todas as nações e culminando na Grande Comissão: "[...] Ide e por todo mundo, pregai o evangelho a toda criatura" (Mc 16:15)\footnote{MARCOS. In: A BÍBLIA SAGRADA: Almeida Corrigida Fiel. São Paulo, 2011.}. Ou seja, Cristo inaugura a participação auxiliar de seus seguidores, que posteriormente seriam a igreja, nessa missão de forma expansiva e externa. Assim, através de uma leitura da Bíblia como a revelação progressiva de Deus, podemos também perceber a progressão da \emph{Missio Dei} na história: Deus em missão para a redenção da humanidade.
\section{ISRAEL COMO CENTRO DA MISSÃO NO ANTIGO TESTAMENTO}
\subsection{ISRAEL: ELEITO PARA TESTEMUNHAR}
Israel, em sua formação como povo, é escolhido por Deus para ser seu "[...] reino de sacerdotes e nação santa" (Êx 19:6)\footnote{ÊXODO. In: A BÍBLIA SAGRADA: Almeida Corrigida Fiel. São Paulo, 2011.} no mundo. A posição de Israel, portanto, é análoga a posição da tribo de Levi entre as doze tribos. O sacerdócio não pressupunha o proselitismo, mas a manutenção do templo, dos ritos e da instrução da Palavra. Eram, portanto, únicos e exclusivos ao acesso direto à presença de Deus. Essa eleição é também exposta por Paulo em sua carta aos Romanos:
\begin{citacao}
Em Cristo digo a verdade, não minto (dando-me testemunho a minha consciência no Espírito Santo): Que tenho grande tristeza e contínua dor no meu coração. Porque eu mesmo poderia desejar ser anátema de Cristo, por amor de meus irmãos, que são meus parentes segundo a carne; \textbf{que são israelitas, dos quais é a adoção de filhos, e a glória, e as alianças, e a lei, e o culto, e as promessas}; dos quais são os pais, e dos quais é Cristo segundo a carne, o qual é sobre todos, Deus bendito eternamente. Amém. (Rm 9:1-5)\footnote{ROMANOS. In: A BÍBLIA SAGRADA: Almeida Corrigida Fiel. São Paulo, 2011.}
\end{citacao}
Logo, assim como um candeeiro em meio a noite que convida os que andam no escuro a se achegar e caminhar sob a luz, a eleição de Israel se dá para que todos os outros povos se acheguem ao Deus verdadeiro. Um pequeno povo, cercado de grandes nações, que o próprio Deus engrandeceu para que sua glória resplandecesse neles para que toda a terra tivesse o testemunho dEle (Sl 67:1-2)\footnote{SALMOS. In: A BÍBLIA SAGRADA: Almeida Corrigida Fiel. São Paulo, 2011.} e que fosse o centro da comunhão da humanidade com Ele como narrado também em Salmos 87.
\subsection{DEUS EM MISSÃO BUSCA TODAS AS NAÇÕES DO MUNDO}
Mesmo com este entendimento da eleição de Israel e a conclusão comum de que temos em enxergar o papel missionário de Israel como um movimento de fora para dentro, não podemos entender que este também é o movimento de Deus na história. O papel de Israel como povo escolhido era, de fato, o sacerdócio, mas Deus esteve em ação direta, de dentro para fora, através dos profetas:
\begin{citacao}
Disse mais: Pouco é que sejas o meu servo, para restaurares as tribos de Jacó, e tornares a trazer os preservados de Israel; \textbf{também te dei para luz dos gentios, para seres a minha salvação até à extremidade da terra}. (Is 49:6)\footnote{ISAÍAS. In: A BÍBLIA SAGRADA: Almeida Corrigida Fiel. São Paulo, 2011.}
\end{citacao}
É, portanto, através dos profetas que Deus se colocou em missão redentiva à todos os povos. Mesmo em face de exílios e cativeiros, Deus se fez presente\footnote{Uma tensão interessante aqui é como os israelitas exilados em terras distantes tinham muita dificuldade em entender que a distância de Jerusalém não significava que estavam distantes de Deus. A visão de Ezequiel acerca do trono de Deus como algo móvel, com rodas como carruagens, as exortações de Jeremias quanto a Deus ser de perto e também de longe trazem tanto o afago e esperança de que Deus não os abandonou mas também apontam para a soberania de Deus acerca de toda a terra. Apesar de Sião ser sua cidade, toda a sua terra é também sua propriedade.} tanto ao seus escolhidos quanto os usou como voz profética para a salvação até mesmo de seus captores. Alguns teólogos vão entender que Israel fracassou em sua vocação e esse fracasso seria evidenciado pelos quatrocentos anos de silêncio profético. Independente da controvérsia gerada em tal afirmação, o fato é que esses quatrocentos anos aconteceram sem nenhuma progressão na Revelação. Pode-se entender esse silêncio como o mundo sendo preparado para a \emph{plenitude dos tempos}, culminando na vinda do Messias: "Mas, vindo a plenitude dos tempos, Deus enviou seu Filho, nascido de mulher, nascido sob a lei, para remir os que estavam debaixo da lei, a fim de recebermos a adoção de filhos." (Gl 4:4-5)\footnote{GÁLATAS. In: A BÍBLIA SAGRADA: Almeida Corrigida Fiel. São Paulo, 2011.}.



\pagebreak
\section{DECLARAÇÃO DE INTEGRIDADE ACADÊMICA}
Eu, Gabriel Cardoso dos Santos Faleiro, declaro que produzi este texto de maneira íntegra e original, sem recorrer ao plágio ou ao uso de inteligência artificial para sua criação. Todas as ideias, argumentos e referências foram desenvolvidos de forma honesta, garantindo que o conteúdo reflita exclusivamente meu próprio raciocínio e pesquisa.
\pagebreak

\nocite{BIBLIA}
\nocite{EDILSON}
\renewcommand{\bibname}{{REFER\^ENCIAS}}
\bibliography{mt1_entrega_01.bib}

\end{document}
