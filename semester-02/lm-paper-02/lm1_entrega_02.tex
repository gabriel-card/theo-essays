
\documentclass[
	% -- opções da classe memoir --
    article,            % artigo academico
	12pt,				% tamanho da fonte
	%openright,			% capítulos começam em pág ímpar (insere página vazia caso preciso)
	oneside,			% para impressão em recto e verso. Oposto a oneside (twoside)
	a4paper,			% tamanho do papel. 
	% -- opções da classe abntex2 --
	chapter=TITLE,		% títulos de capítulos convertidos em letras maiúsculas
	section=TITLE,		% títulos de seções convertidos em letras maiúsculas
	%subsection=TITLE,	% títulos de subseções convertidos em letras maiúsculas
	%subsubsection=TITLE,% títulos de subsubseções convertidos em letras maiúsculas
	% -- opções do pacote babel --
	english,			% idioma adicional para hifenização
	french,				% idioma adicional para hifenização
	spanish,			% idioma adicional para hifenização
	brazil				% o último idioma é o principal do documento
	]{abntex2}

% ---
% Pacotes básicos 
% ---
\usepackage{times}				% Usa a fonte Times Roman			
\usepackage[T1]{fontenc}		% Selecao de codigos de fonte.
\usepackage[utf8]{inputenc}		% Codificacao do documento (conversão automática dos acentos)
\usepackage{indentfirst}		% Indenta o primeiro parágrafo de cada seção.
\usepackage{color}				% Controle das cores
\usepackage{graphicx}			% Inclusão de gráficos
\usepackage{microtype} 			% para melhorias de justificação
% ---

% ---
% Pacotes de citações
% ---
\usepackage[brazilian,hyperpageref]{backref}	 % Paginas com as citações na bibl
\usepackage[alf]{abntex2cite}	% Citações padrão ABNT

% --- 
% CONFIGURAÇÕES DE PACOTES
% --- 

% ---
% Configurações do pacote backref
% Usado sem a opção hyperpageref de backref
\renewcommand{\backrefpagesname}{Citado na(s) página(s):~}
% Texto padrão antes do número das páginas
\renewcommand{\backref}{}
% Define os textos da citação
\renewcommand*{\backrefalt}[4]{
	\ifcase #1 %
		Nenhuma citação no texto.%
	\or
		Citado na página #2.%
	\else
		Citado #1 vezes nas páginas #2.%
	\fi}%
% ---
% ---
% FORMATAÇAO FLAM
% ---
\setlength{\parindent}{1.25cm}
\setlength{\parskip}{0.5cm}
\setlength\afterchapskip{\lineskip}
\setlrmarginsandblock{3cm}{2cm}{*}
\setulmarginsandblock{3cm}{2cm}{*}
\checkandfixthelayout
\renewcommand{\ABNTEXchapterfont}{\normalfont}
\renewcommand{\ABNTEXsectionfontsize}{\large\bfseries}
\renewcommand{\cftsectionfont}{\bfseries\MakeTextUppercase}
\renewcommand{\ABNTEXsubsectionfontsize}{\normalsize}
\renewcommand{\cftsubsectionfont}{\normalfont\MakeTextUppercase} % Tirar negrito das subsecoes no sumario
\renewcommand{\ABNTEXsubsubsectionfontsize}{\normalsize\bfseries}
\renewcommand{\cftsubsubsectionfont}{\bfseries} % Tirar negrito das subsecoes no sumario
% ---
% Informações de dados para CAPA e FOLHA DE ROSTO
% ---
\titulo{LIDERANÇA E MINISTÉRIO \\ ENTREGA 2}
\autor{GABRIEL CARDOSO DOS SANTOS FALEIRO}
\local{ARUJÁ-SP}
\data{2025}
\instituicao{%
  FLAM - FACULDADE LATINO AMERICANA
}
\tipotrabalho{ENTREGA 1}
% O preambulo deve conter o tipo do trabalho, o objetivo, 
% o nome da instituição e a área de concentração 
\preambulo{Trabalho da disciplina de Liderança e Ministério, solicitado pelo prof. José Januário da Silva Filho}
% ---


% ---
% Configurações de aparência do PDF final

% alterando o aspecto da cor azul
\definecolor{blue}{RGB}{41,5,195}

% informações do PDF
\makeatletter
\hypersetup{
     	%pagebackref=true,
		pdftitle={\@title}, 
		pdfauthor={\@author},
    	pdfsubject={\imprimirpreambulo},
	    pdfcreator={Gabriel Cardoso dos Santos Faleiro},
		pdfkeywords={abnt}{latex}{abntex}{abntex2}{trabalho acadêmico},
		hidelinks=true,
		% colorlinks=true,       		% false: boxed links; true: colored links
    	linkcolor=blue,          	% color of internal links
    	citecolor=blue,        		% color of links to bibliography
    	filecolor=magenta,      		% color of file links
		urlcolor=blue,
		bookmarksdepth=4
}
\makeatother
% ---
% ---
% compila o indice
% ---
\makeindex
% ---

% ----
% Início do documento
% ----
\begin{document}

\citeoption{abnt-full-initials=yes}


% Seleciona o idioma do documento (conforme pacotes do babel)
%\selectlanguage{english}
\selectlanguage{brazil}
% ----------------------------------------------------------
% ELEMENTOS PRÉ-TEXTUAIS
% ----------------------------------------------------------
% \pretextual

% ---
% Capa
% ---
\renewcommand{\imprimircapa}{%
  \begin{capa}%
    \center
    \ABNTEXchapterfont\large\imprimirinstituicao

    \ABNTEXchapterfont\large\imprimirautor

    \vfill
    \begin{center}
    \ABNTEXchapterfont\bfseries\large\imprimirtitulo
    \end{center}
    \vfill

    \large\imprimirlocal %<<<<<<<<<<<mude

    \large\imprimirdata %<<<<<<<< mude

    \vspace*{1cm}
  \end{capa}
}
\imprimircapa
% ---

% ---
% Folha de rosto
% (o * indica que haverá a ficha bibliográfica)
% ---
% \imprimirfolhaderosto*
\imprimirfolhaderosto
% ---

% ---
% inserir o sumario
% ---
\pdfbookmark[0]{\contentsname}{toc}
\tableofcontents*
\cleardoublepage
% ---

% ----------------------------------------------------------
% ELEMENTOS TEXTUAIS
% ----------------------------------------------------------
\textual
\pagestyle{simple}

% ----------------------------------------------------------
% Introdução (mas presente no Sumário)

% Entrega 2 - Observando com atenção o Texto Bíblico de João 10: 1 a 14, descreva o conceito de “Liderança por Pastoreio “ exemplificando na conduta ministerial de Jesus. Mínimo: 01 página.                                    


\section{A LIDERANÇA, O PASTOR E A PORTA}
Encontramos no décimo capítulo do evangelho de João um breve sermão de Cristo acerca de sua identidade e da identidade de seus seguidores e discípulos. Se aproveitando de elementos presentes na cultura judaica da época, Jesus utiliza as figuras relacionadas a um rebanho: ovelhas, pastor, curral. No decorrer do texto, pode-se captar algumas nuâncias que exemplificam o que se chama de liderança por pastoreio.

Cristo cria uma distinção clara entre Ele e falsos mestres que o antecederam: as ovelhas não reconheceram as vozes desses falsos mestres, a quem chamou de ladrões e salteadores. Como não a reconheceram, não o seguiram; diferentemente dEle em que as ovelhas reconheceram sua voz e o seguiram, se chamando assim de pastor. Esse reconhecimento por parte das ovelhas não se dá num vácuo: a imagem de um pastor de ovelhas evoca uma função de guia e cuidado constante de um rebanho inteiro. As ovelhas reconhecem a voz de seu pastor porque estão com seu pastor todos os dias, sendo guiadas, alimentadas, protegidas e cuidadas. Ou seja, da mesma forma que o serviço prestado pelo pastor das ovelhas tornam-o reconhecido pelas ovelhas como seu pastor, um líder pastoral também só será reconhecido como líder pelos seus liderados mediante seu serviço prestado.

Temos, assim, a subversão da expectativa do que é um líder pela secularidade e o líder pastoral apresentado por Cristo. Em contraste a líderes autocráticos, militarescos e inflamados, temos a figura de alguém que dedica seu tempo, esforço físico e até mesmo sua própria segurança em prol de seus liderados. Porém existe uma imagem que Cristo também atribui a si que distoa da figura pastoral que ele apresenta: a porta. Cristo se personifica na porta do aprisco dizendo que Ele é a porta das ovelhas; e completa dizendo que quem entrar por Ele será salvo e encontrará pasto. Ao se colocar tanto na figura do pastor quanto da porta do aprisco, Cristo está dizendo que Ele não é apenas um líder, mas também o Messias que restaurará todas as coisas.

Assim, o trabalho de um pastor ou líder ministerial no exercício de sua liderança precisa sempre apontar para a porta, que é Cristo. Servir à sua comunidade com zelo, amor e, sobretudo, guiando-os na direção da salvação. Da mesma forma que um pastor cuida do rebanho inteiro ao mesmo tempo que se ausenta para resgatar uma única ovelha, é necessário que enquanto se guia toda a comunidade em discipulado, o líder se atente às necessidades individuais de cada liderado para que consigam caminhar juntos e, por fim, adentrar a porta e à salvação prometida por Jesus.


\pagebreak
\section{DECLARAÇÃO DE INTEGRIDADE ACADÊMICA}
Eu, Gabriel Cardoso dos Santos Faleiro, declaro que produzi este texto de maneira íntegra e original, sem recorrer ao plágio ou ao uso de inteligência artificial para sua criação. Todas as ideias, argumentos e referências foram desenvolvidos de forma honesta, garantindo que o conteúdo reflita exclusivamente meu próprio raciocínio e pesquisa.
% ----------------------------------------------------------

\pagebreak
\renewcommand{\bibname}{{REFER\^ENCIAS}}
\bibliographystyle{abntex2-alf}
\bibliography{at2_entrega_01.bib}

\end{document}
