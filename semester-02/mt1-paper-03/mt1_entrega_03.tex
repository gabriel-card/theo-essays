
\documentclass[
	% -- opções da classe memoir --
    article,            % artigo academico
	12pt,				% tamanho da fonte
	%openright,			% capítulos começam em pág ímpar (insere página vazia caso preciso)
	oneside,			% para impressão em recto e verso. Oposto a oneside (twoside)
	a4paper,			% tamanho do papel. 
	% -- opções da classe abntex2 --
	chapter=TITLE,		% títulos de capítulos convertidos em letras maiúsculas
	section=TITLE,		% títulos de seções convertidos em letras maiúsculas
	%subsection=TITLE,	% títulos de subseções convertidos em letras maiúsculas
	%subsubsection=TITLE,% títulos de subsubseções convertidos em letras maiúsculas
	% -- opções do pacote babel --
	english,			% idioma adicional para hifenização
	french,				% idioma adicional para hifenização
	spanish,			% idioma adicional para hifenização
	brazil				% o último idioma é o principal do documento
	]{abntex2}

% ---
% Pacotes básicos 
% ---
\usepackage{times}				% Usa a fonte Times Roman			
\usepackage[T1]{fontenc}		% Selecao de codigos de fonte.
\usepackage[utf8]{inputenc}		% Codificacao do documento (conversão automática dos acentos)
\usepackage{indentfirst}		% Indenta o primeiro parágrafo de cada seção.
\usepackage{color}				% Controle das cores
\usepackage{graphicx}			% Inclusão de gráficos
\usepackage{microtype} 			% para melhorias de justificação
% ---

% ---
% Pacotes de citações
% ---
\usepackage[brazilian,hyperpageref]{backref}	 % Paginas com as citações na bibl
\usepackage[alf]{abntex2cite}	% Citações padrão ABNT

% --- 
% CONFIGURAÇÕES DE PACOTES
% --- 

% ---
% Configurações do pacote backref
% Usado sem a opção hyperpageref de backref
\renewcommand{\backrefpagesname}{Citado na(s) página(s):~}
% Texto padrão antes do número das páginas
\renewcommand{\backref}{}
% Define os textos da citação
\renewcommand*{\backrefalt}[4]{}%
% ---
% ---
% FORMATAÇAO FLAM
% ---
\setlength{\parindent}{1.25cm}
\setlength{\parskip}{0.5cm}
\setlength\afterchapskip{\lineskip}
\setlrmarginsandblock{3cm}{2cm}{*}
\setulmarginsandblock{3cm}{2cm}{*}
\checkandfixthelayout
\renewcommand{\ABNTEXchapterfont}{\normalfont}
\renewcommand{\ABNTEXsectionfontsize}{\large\bfseries}
\renewcommand{\cftsectionfont}{\bfseries\MakeTextUppercase}
\renewcommand{\ABNTEXsubsectionfontsize}{\normalsize}
\renewcommand{\cftsubsectionfont}{\normalfont\MakeTextUppercase} % Tirar negrito das subsecoes no sumario
\renewcommand{\ABNTEXsubsubsectionfontsize}{\normalsize\bfseries}
\renewcommand{\cftsubsubsectionfont}{\bfseries} % Tirar negrito das subsecoes no sumario
% ---
% Informações de dados para CAPA e FOLHA DE ROSTO
% ---
\titulo{MISSÕES TRANSCULTURAIS \\ ENTREGA 2}
\autor{GABRIEL CARDOSO DOS SANTOS FALEIRO}
\local{ARUJÁ-SP}
\data{2025}
\instituicao{%
  FLAM - FACULDADE LATINO AMERICANA
}
\tipotrabalho{ENTREGA 2}
% O preambulo deve conter o tipo do trabalho, o objetivo, 
% o nome da instituição e a área de concentração 
\preambulo{Trabalho da disciplina de Missões Transculturais, solicitado pelo prof. Dr. Edilson Botelho Nogueira.}
% ---


% ---
% Configurações de aparência do PDF final

% alterando o aspecto da cor azul
\definecolor{blue}{RGB}{41,5,195}

% informações do PDF
\makeatletter
\hypersetup{
     	%pagebackref=true,
		pdftitle={\@title}, 
		pdfauthor={\@author},
    	pdfsubject={\imprimirpreambulo},
	    pdfcreator={Gabriel Cardoso dos Santos Faleiro},
		pdfkeywords={abnt}{latex}{abntex}{abntex2}{trabalho acadêmico}, 
		colorlinks=true,       		% false: boxed links; true: colored links
    	linkcolor=blue,          	% color of internal links
    	citecolor=blue,        		% color of links to bibliography
    	filecolor=magenta,      		% color of file links
		urlcolor=blue,
		bookmarksdepth=4
}
\makeatother
% ---
% ---
% compila o indice
% ---
\makeindex
% ---

% ----
% Início do documento
% ----
\begin{document}

\citeoption{abnt-full-initials=yes}


% Seleciona o idioma do documento (conforme pacotes do babel)
%\selectlanguage{english}
\selectlanguage{brazil}
% ----------------------------------------------------------
% ELEMENTOS PRÉ-TEXTUAIS
% ----------------------------------------------------------
% \pretextual

% ---
% Capa
% ---
\renewcommand{\imprimircapa}{%
  \begin{capa}%
    \center
    \ABNTEXchapterfont\large\imprimirinstituicao

    \ABNTEXchapterfont\large\imprimirautor

    \vfill
    \begin{center}
    \ABNTEXchapterfont\bfseries\large\imprimirtitulo
    \end{center}
    \vfill

    \large\imprimirlocal %<<<<<<<<<<<mude

    \large\imprimirdata %<<<<<<<< mude

    \vspace*{1cm}
  \end{capa}
}
\imprimircapa
% ---

% ---
% Folha de rosto
% (o * indica que haverá a ficha bibliográfica)
% ---
% \imprimirfolhaderosto*
\imprimirfolhaderosto
% ---

% ---
% inserir o sumario
% ---
\pdfbookmark[0]{\contentsname}{toc}
\tableofcontents*
\cleardoublepage
% ---

% ----------------------------------------------------------
% ELEMENTOS TEXTUAIS
% ----------------------------------------------------------
\textual
\pagestyle{simple}

% ----------------------------------------------------------
% Introdução (mas presente no Sumário)

% INTRODUÇÃO
% ● Apresentação do Tema: O que é a Missio Dei?
% 1. ISRAEL COMO CENTRO DA MISSÃO NO ANTIGO TESTAMENTO
% 1.1. ISRAEL: ELEITO PARA TESTEMUNHAR
% ● A Eleição de Israel como Reino de Sacerdotes, Êxodo 19:5–6
% ● A vocação missionária de Israel, Salmo 67
% ● Jerusalém como centro redentivo para o mundo Salmo 87
% 1. 2. DEUS EM MISSÃO BUSCA TODAS AS NAÇÕES DO MUNDO
% ● A mensagem dos Profetas aos outros povos da terra.
% ● O Cativeiro de Israel como oportunidade missional
% ● O Silencio missional de Deus e seu preparo para a Plenitude do Tempo

\section{INTRODUÇÃO}
\emph{Missio Dei}\footnote{Em tradução livre: Missão de Deus.}, termo criado pelo teólogo alemão Karl Hartenstein\footnote{Karl Hartenstein (1894-1952) foi um teólogo alemão que cunhou o termo pela primeira vez para distinguir entre a missão de Deus e a missão da Igreja (\emph{missio ecclesiae}), no artigo \emph{Wozu nötigt die Finanzlage der Mission} contido na revista \emph{Revista Evangelisches Missions} volume 79.}, carrega consigo uma mudança significativa que acontecia em sua época na perspectiva missiológica da cristandade. Hartenstein argumenta que a missão da Igreja não parte da humanidade, mas sim de uma ação direta de Deus na história. Essa ação contínua está presente em toda a narrativa bíblica, passando pela escolha de Israel como povo de Deus para que testemunhassem sobre Ele para todas as nações e culminando na Grande Comissão: "[...] Ide e por todo mundo, pregai o evangelho a toda criatura" (Mc 16:15)\footnote{MARCOS. In: A BÍBLIA SAGRADA: Almeida Corrigida Fiel. São Paulo, 2011.}. Ou seja, Cristo inaugura a participação auxiliar de seus seguidores, que posteriormente seriam a igreja, nessa missão de forma expansiva e externa. Assim, através de uma leitura da Bíblia como a revelação progressiva de Deus, podemos também perceber a progressão da \emph{Missio Dei} na história: Deus em missão para a redenção da humanidade.
\section{ISRAEL COMO CENTRO DA MISSÃO NO ANTIGO TESTAMENTO}
\subsection{ISRAEL: ELEITO PARA TESTEMUNHAR}
Israel, em sua formação como povo, é escolhido por Deus para ser seu "[...] reino de sacerdotes e nação santa" (Êx 19:6)\footnote{ÊXODO. In: A BÍBLIA SAGRADA: Almeida Corrigida Fiel. São Paulo, 2011.} no mundo. A posição de Israel, portanto, é análoga a posição da tribo de Levi entre as doze tribos. O sacerdócio não pressupunha o proselitismo, mas a manutenção do templo, dos ritos e da instrução da Palavra. Eram, portanto, únicos e exclusivos ao acesso direto à presença de Deus. Essa eleição é também exposta por Paulo em sua carta aos Romanos:
\begin{citacao}
Em Cristo digo a verdade, não minto (dando-me testemunho a minha consciência no Espírito Santo): Que tenho grande tristeza e contínua dor no meu coração. Porque eu mesmo poderia desejar ser anátema de Cristo, por amor de meus irmãos, que são meus parentes segundo a carne; \textbf{que são israelitas, dos quais é a adoção de filhos, e a glória, e as alianças, e a lei, e o culto, e as promessas}; dos quais são os pais, e dos quais é Cristo segundo a carne, o qual é sobre todos, Deus bendito eternamente. Amém. (Rm 9:1-5)\footnote{ROMANOS. In: A BÍBLIA SAGRADA: Almeida Corrigida Fiel. São Paulo, 2011.}
\end{citacao}
Logo, assim como um candeeiro em meio a noite que convida os que andam no escuro a se achegar e caminhar sob a luz, a eleição de Israel se dá para que todos os outros povos se acheguem ao Deus verdadeiro. Um pequeno povo, cercado de grandes nações, que o próprio Deus engrandeceu para que sua glória resplandecesse neles para que toda a terra tivesse o testemunho dEle (Sl 67:1-2)\footnote{SALMOS. In: A BÍBLIA SAGRADA: Almeida Corrigida Fiel. São Paulo, 2011.} e que fosse o centro da comunhão da humanidade com Ele como narrado também em Salmos 87.
\subsection{DEUS EM MISSÃO BUSCA TODAS AS NAÇÕES DO MUNDO}
Mesmo com este entendimento da eleição de Israel e a conclusão comum de que temos em enxergar o papel missionário de Israel como um movimento de fora para dentro, não podemos entender que este também é o movimento de Deus na história. O papel de Israel como povo escolhido era, de fato, o sacerdócio, mas Deus esteve em ação direta, de dentro para fora, através dos profetas:
\begin{citacao}
Disse mais: Pouco é que sejas o meu servo, para restaurares as tribos de Jacó, e tornares a trazer os preservados de Israel; \textbf{também te dei para luz dos gentios, para seres a minha salvação até à extremidade da terra}. (Is 49:6)\footnote{ISAÍAS. In: A BÍBLIA SAGRADA: Almeida Corrigida Fiel. São Paulo, 2011.}
\end{citacao}
É, portanto, através dos profetas que Deus se colocou em missão redentiva à todos os povos. Mesmo em face de exílios e cativeiros, Deus se fez presente\footnote{Uma tensão interessante aqui é como os israelitas exilados em terras distantes tinham muita dificuldade em entender que a distância de Jerusalém não significava que estavam distantes de Deus. A visão de Ezequiel acerca do trono de Deus como algo móvel, com rodas como carruagens, as exortações de Jeremias quanto a Deus ser de perto e também de longe trazem tanto o afago e esperança de que Deus não os abandonou mas também apontam para a soberania de Deus acerca de toda a terra. Apesar de Sião ser sua cidade, toda a Terra também é sua propriedade.} tanto ao seus escolhidos quanto os usou como voz profética para a salvação até mesmo de seus captores. Alguns teólogos vão entender que Israel fracassou em sua vocação e esse fracasso seria evidenciado pelos quatrocentos anos de silêncio profético. Independente da controvérsia gerada em tal afirmação, o fato é que esses quatrocentos anos aconteceram sem nenhuma progressão na Revelação. Pode-se entender esse silêncio como o mundo sendo preparado para a \emph{plenitude dos tempos}, culminando na vinda do Messias: "Mas, vindo a plenitude dos tempos, Deus enviou seu Filho, nascido de mulher, nascido sob a lei, para remir os que estavam debaixo da lei, a fim de recebermos a adoção de filhos." (Gl 4:4-5)\footnote{GÁLATAS. In: A BÍBLIA SAGRADA: Almeida Corrigida Fiel. São Paulo, 2011.}.
\section{A IGREJA COMO O ISRAEL DE DEUS NO NOVO TESTAMENTO}
\subsection{A MISSIO DEI NO NOVO TESTAMENTO}
% ● A Igreja como herdeira do sacerdócio real (1 Pedro 2:9)
Em congruência com seu modus operandi por toda a Antiga Aliança, temos no Novo Testamento o mesmo papel de \emph{quem envia} e de \emph{quem prepara} exercidos por Deus. Acerca do papel de \emph{quem envia}; Deus enviou seu Filho, o Cristo, para: "[...] evangelizar os pobres, [...] curar os contritos de coração, a proclamar liberdade aos cativos, e restauração da vista aos cegos, a pôr em liberdade os oprimidos, a anunciar o ano aceitável do Senhor." (Lc 4:18-19)\footnote{LUCAS. In: A BÍBLIA SAGRADA: Almeida Corrigida Fiel. São Paulo, 2011.}. O ponta-pé inicial para o ministério messiânico de Cristo passa pelo cumprimento desta profecia de Isaías e é através deste ministério de libertação que a revelação da Missio Dei progride na inauguração do Reino de Deus.

Diferente da Antiga Aliança onde um único povo detinha a revelação de Deus, temos o início de um Reinado que transpassa por qualquer barreira étnica e política, onde judeus e gentios são chamados para se tornar um novo e único povo, distintos de qualquer marca anterior. Paulo em sua segunda epístola aos Coríntios demonstra como, no sacrifício de Cristo, não há mais distinção entre os povos: "E ele morreu \textbf{por todos}, para que os que vivem não vivam mais para si, mas para aquele que por eles morreu e ressuscitou." (2Co 5:15)\footnote{2 CORÍNTIOS. In: A BÍBLIA SAGRADA: Almeida Corrigida Fiel. São Paulo, 2011.}, e que nesta nova vida todos são novas criaturas: "[...] se alguém está em Cristo, nova criatura é; as coisas velhas já passaram; eis que tudo se fez novo" (2Co 5:17)\footnote{2 CORÍNTIOS. In: A BÍBLIA SAGRADA: Almeida Corrigida Fiel. São Paulo, 2011.}. Assim, com a inauguração do Reino também há a criação de um novo povo para servir em sacerdócio: a Igreja.

% ● A Missio Dei tem uma Igreja (Mateus 28:19–20)
Este sacerdócio se difere do papel de Israel, pressupondo um movimento da própria Igreja a todos os povos ainda não alcançados. É na Grande Comissão, descrito no evangelho de Mateus, que temos Cristo encarregando a Igreja da proclamação de seu ministério e fazer discípulos:
\begin{citacao}
"Portanto ide, fazei discípulos de todas as nações, batizando-os em nome do Pai, e do Filho, e do Espírito Santo; ensinando-os a guardar todas as coisas que eu vos tenho mandado; e eis que eu estou convosco todos os dias, até a consumação do mundo. Amém." (Mt 28:19-20)
\end{citacao}
E, também, é no cumprimento desse mandato pela Igreja que se nota o resultado do trabalho de preparo de Deus na Missio Dei. Os apóstolos foram enviados para colher o que eles não plantaram, ou seja, proclamar o Evangelho para quem já havia sido preparado para recebê-lo. Analogia esta proposta por Jesus, demonstrando aos seus discípulos o anseio pelo Evangelho por parte dos samaritanos: "Eu vos enviei a ceifar onde vós não trabalhastes; outros trabalharam, e vós entrastes no seu trabalho." (Jo 4:38). Ora, quem mais poderia ter gerado o anseio dos samaritanos, não havendo profetas por centenas de anos, senão o próprio Deus?

Assim, temos a figura de um Deus missionário que prepara os povos para receber sua revelação e que envia sua Igreja para proclamar essa revelação. O papel da Igreja, portanto, é auxiliar e não protagonista; a responsabilidade da proclamação é de extrema seriedade, mas não torna a Igreja o centro da missão de Deus para a redenção da humanidade.

\subsection{MOVIMENTO CENTRÍFUGO}
Retomando a analogia do candeeiro em meio a noite, onde os povos se achegavam a Israel como os que andam no escuro se achegavam a luz, temos na Nova Aliança um movimento diferente: o candeeiro agora segue rumo a escuridão e acende outros candeeiros pelo caminho, e estes também seguem rumo a escuridão por todas as direções e acendem mais candeeiros. Deus envia seu Filho, o Cristo; Cristo envia os apóstolos e todos os mais que o ouviam na Grande Comissão. Podemos notar esse movimento de expansão, ou centrífugo, nas palavras de Jesus pouco antes de ascender aos céus, no primeiro capítulo de Atos onde há uma clara noção de expansão da área de impacto do testemunho dos apóstolos: "Mas recebereis o poder do Espírito Santo, que há de vir sobre vós; e ser-me-eis testemunhas, tanto em Jerusalém como em toda a Judeia e Samaria, e até aos confins da terra." (Atos 1:8)

Assim, nesse papel auxiliador da Igreja, é esperado que a proclamação do Evangelho seja feita para todo povo em toda língua, até o retorno prometido por Cristo. Temos diante dessa recém-nascida Igreja, então, um grande desafio: superar diferenças culturais, étnicas e até linguísticas para que a mensagem de Cristo seja entregue a todo povo. Encontra-se no livro de Atos soluções sobrenaturais para alguns dos problemas, como o dom de línguas onde os apóstolos passaram a proclamar o Evangelho em línguas estrangeiras sem nunca tê-las aprendido e, até mesmo, arrebatamentos como o de Filipe. Mas nem todas soluções que o Espírito Santo manifestou foram miraculosas: desde o início foi necessário a contextualização dos próprios apóstolos aos povos que visitavam para que eles fossem recebidos e ouvidos. Paulo em sua primeira epístola aos Coríntios sintetiza esse esforço:
\begin{citacao}
E fiz-me como judeu para os judeus, para ganhar os judeus; para os que estão debaixo da lei, como se estivesse debaixo da lei, para ganhar os que estão debaixo da lei. [...] Fiz-me como fraco para os fracos, para ganhar os fracos. Fiz-me tudo para todos, para por todos os meios chegar a salvar alguns. E eu faço isto por causa do evangelho, para ser também participante dele. (1Co 9:20,22-23)
\end{citacao}

Essa contextualização se dá em diversas esferas, tais como Whiteman apresenta: a tradução das Escrituras, a linguagem (não apenas a língua) do missionário, a forma de evangelismo, liturgia e educação teológica. Ou seja, o papel da Igreja exige um esforço intelectual e empático que procura se tornar o mais inteligível possível para um povo enquanto a essência da mensagem continua intacta. Essência essa que é ofensiva para a natureza caída humana, como Paulo lembra na primeira epístola aos Coríntios: "Mas nós pregamos a Cristo crucificado, que é escândalo para os judeus, e loucura para os gregos." (1Co 1:23) É, então, na adversariedade inicial dessa mensagem contra a natureza caída que temos o objetivo da proclamação do Evangelho. Neste momento de escândalo nasce o arrependimento e, desse arrependimento, a restauração do relacionamento de Deus com a humanidade.

\section{CONCLUSÃO}
A missão da Igreja, portanto, está contida na missão de Deus. É através do envio contínuo de si mesma ao mundo que se cumpre o papel auxiliador: o meio ao qual a mensagem do Evangelho é transmitida a todas as gerações de todas as nações. Neste envio se encontram desafios como a contextualização necessária para que a mensagem seja inteligível para um determinado povo sob uma determinada cultura, cuja dificuldade se encontra tanto em termos intelectuais como proclamar o Evangelho ou traduzir as Escrituras em uma nova língua quanto em termos práticos como transitar em boa permeabilidade e se fazer presente entre diferentes círculos econômicos e sociais.

Em outros termos, a contextualização depende não apenas do evangelismo (a proclamação falada do Evangelho) mas também toda atividade missiológica que a Igreja exerce em uma comunidade que comunica a chegada do Reino de Deus, que se torna palpável pelo cumprimento da profecia de Isaías por Cristo: "O Espírito do Senhor está sobre mim, pelo que me ungiu para evangelizar os pobres; enviou-me para proclamar libertação aos cativos e restauração da vista aos cegos, para pôr em liberdade os oprimidos, e apregoar o ano aceitável do Senhor." (Lc 4:18-19) A materialidade dessa profecia, hoje, também depende do trabalho missiológico da Igreja através das ações sociais respondendo a necessidades humanas imediatas.
\begin{citacao}
Não há evangelização sem solidariedade; não existe solidariedade cristã que não implique compartilhar o conhecimento do reino que é a promessa de Deus aos pobres da terra. Ocorre aqui um duplo teste de credibilidade: uma proclamação que não exiba as promessas de justiça do reino aos pobres da terra caricaturiza o evangelho, mas uma participação cristã nas lutas por justiça que não aponte para as promessas do reino também torna caricatural a compreensão cristã de justiça. \cite[p.488]{BOSCH}
\end{citacao}

\pagebreak
\section{DECLARAÇÃO DE INTEGRIDADE ACADÊMICA}
Eu, Gabriel Cardoso dos Santos Faleiro, declaro que produzi este texto de maneira íntegra e original, sem recorrer ao plágio ou ao uso de inteligência artificial para sua criação. Todas as ideias, argumentos e referências foram desenvolvidos de forma honesta, garantindo que o conteúdo reflita exclusivamente meu próprio raciocínio e pesquisa.
\pagebreak

\nocite{BIBLIA}
\nocite{EDILSON}
\renewcommand{\bibname}{{REFER\^ENCIAS}}
\bibliography{mt1_entrega_03.bib}

\end{document}
