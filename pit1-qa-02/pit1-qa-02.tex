
\documentclass[
	% -- opções da classe memoir --
    article,            % artigo academico
	12pt,				% tamanho da fonte
	%openright,			% capítulos começam em pág ímpar (insere página vazia caso preciso)
	oneside,			% para impressão em recto e verso. Oposto a oneside (twoside)
	a4paper,			% tamanho do papel. 
	% -- opções da classe abntex2 --
	chapter=TITLE,		% títulos de capítulos convertidos em letras maiúsculas
	section=TITLE,		% títulos de seções convertidos em letras maiúsculas
	%subsection=TITLE,	% títulos de subseções convertidos em letras maiúsculas
	%subsubsection=TITLE,% títulos de subsubseções convertidos em letras maiúsculas
	% -- opções do pacote babel --
	english,			% idioma adicional para hifenização
	french,				% idioma adicional para hifenização
	spanish,			% idioma adicional para hifenização
	brazil				% o último idioma é o principal do documento
	]{abntex2}

% ---
% Pacotes básicos 
% ---
\usepackage{times}				% Usa a fonte Times Roman			
\usepackage[T1]{fontenc}		% Selecao de codigos de fonte.
\usepackage[utf8]{inputenc}		% Codificacao do documento (conversão automática dos acentos)
\usepackage{indentfirst}		% Indenta o primeiro parágrafo de cada seção.
\usepackage{color}				% Controle das cores
\usepackage{graphicx}			% Inclusão de gráficos
\usepackage{microtype} 			% para melhorias de justificação
% ---

% ---
% Pacotes de citações
% ---
\usepackage[brazilian,hyperpageref]{backref}	 % Paginas com as citações na bibl
\usepackage[alf]{abntex2cite}	% Citações padrão ABNT

% --- 
% CONFIGURAÇÕES DE PACOTES
% --- 

% ---
% Configurações do pacote backref
% Usado sem a opção hyperpageref de backref
\renewcommand{\backrefpagesname}{Citado na(s) página(s):~}
% Texto padrão antes do número das páginas
\renewcommand{\backref}{}
% Define os textos da citação
\renewcommand*{\backrefalt}[4]{
	\ifcase #1 %
		Nenhuma citação no texto.%
	\or
		Citado na página #2.%
	\else
		Citado #1 vezes nas páginas #2.%
	\fi}%
% ---
% ---
% FORMATAÇAO FLAM
% ---
\setlength{\parindent}{1.25cm}
\setlength{\parskip}{0.5cm}
\setlength\afterchapskip{\lineskip}
\setlrmarginsandblock{3cm}{2cm}{*}
\setulmarginsandblock{3cm}{2cm}{*}
\checkandfixthelayout
\renewcommand{\ABNTEXchapterfont}{\normalfont}
\renewcommand{\ABNTEXsectionfontsize}{\large\bfseries}
\renewcommand{\cftsectionfont}{\bfseries\MakeTextUppercase}
\renewcommand{\ABNTEXsubsectionfontsize}{\normalsize}
\renewcommand{\cftsubsectionfont}{\normalfont\MakeTextUppercase} % Tirar negrito das subsecoes no sumario
\renewcommand{\ABNTEXsubsubsectionfontsize}{\normalsize\bfseries}
\renewcommand{\cftsubsubsectionfont}{\bfseries} % Tirar negrito das subsecoes no sumario
% ---
% Informações de dados para CAPA e FOLHA DE ROSTO
% ---
\titulo{PRODUÇÃO E INTERPRETAÇÃO DE TEXTOS \\ QUESTÃO ABERTA 02}
\autor{GABRIEL CARDOSO DOS SANTOS FALEIRO}
\local{ARUJÁ-SP}
\data{2024}
\instituicao{%
  FLAM - FACULDADE LATINO AMERICANA
}
\tipotrabalho{QUESTÃO ABERTA 01}
% O preambulo deve conter o tipo do trabalho, o objetivo, 
% o nome da instituição e a área de concentração 
\preambulo{Trabalho da disciplina de Produção e Interpretação de Textos, solicitado pela Profa. Dra. Inês Murad.}
% ---


% ---
% Configurações de aparência do PDF final

% alterando o aspecto da cor azul
\definecolor{blue}{RGB}{41,5,195}

% informações do PDF
\makeatletter
\hypersetup{
     	%pagebackref=true,
		pdftitle={\@title}, 
		pdfauthor={\@author},
    	pdfsubject={\imprimirpreambulo},
	    pdfcreator={LaTeX with abnTeX2},
		pdfkeywords={abnt}{latex}{abntex}{abntex2}{trabalho acadêmico}, 
		colorlinks=true,       		% false: boxed links; true: colored links
    	linkcolor=blue,          	% color of internal links
    	citecolor=blue,        		% color of links to bibliography
    	filecolor=magenta,      		% color of file links
		urlcolor=blue,
		bookmarksdepth=4
}
\makeatother
% ---
% ---
% compila o indice
% ---
\makeindex
% ---

% ----
% Início do documento
% ----
\begin{document}

\citeoption{abnt-full-initials=yes}


% Seleciona o idioma do documento (conforme pacotes do babel)
%\selectlanguage{english}
\selectlanguage{brazil}
% ----------------------------------------------------------
% ELEMENTOS PRÉ-TEXTUAIS
% ----------------------------------------------------------
% \pretextual

% ---
% Capa
% ---
\imprimircapa
% ---

% ---
% Folha de rosto
% (o * indica que haverá a ficha bibliográfica)
% ---
% \imprimirfolhaderosto*
\imprimirfolhaderosto
% ---

% ---
% inserir o sumario
% ---
% \pdfbookmark[0]{\contentsname}{toc}
% \tableofcontents*
% \cleardoublepage
% ---

% ----------------------------------------------------------
% ELEMENTOS TEXTUAIS
% ----------------------------------------------------------
\textual
\pagestyle{simple}

% ----------------------------------------------------------
% Introdução (mas presente no Sumário)

\section*{CIDADANIA É PARA TODOS}
% ----------------------------------------------------------

A comunidade \emph{queer}\footnote{Utilizando a forma mais ampla de \emph{queer}, que engloba homossexuais, bissexuais, transsexuais, ou qualquer rotulação não-heterossexual.} deve gozar de todos os direitos constitucionais previstos a um cidadão brasileiro. A luta da comunidade \emph{queer} na reivindicação desses direitos é legítima, dado que nenhuma das suas reivindicações legais pressupõe a imposição de qualquer rito às instituições eclesiásticas, inclusive o casamento. Para entendermos que não existe incompatibilidade nem contradição na defesa de seus direitos e a existência de doutrinas cristãs que denunciam a homossexualidade como pecado, precisamos denotar as diferenças entre o casamento como unidade civil e o casamento como mandato divino, as bases protestantes quanto a separação da igreja e do estado e também o princípio norteador de igualdade e justiça perante a lei.

Historicamente, o fenômeno do casamento é observado em diversas nações e culturas que não possuem raízes hebraicas. Esta união costumeiramente tinha como por objetivos a produção de herdeiros legítimos às posses ou cargos sociais e a aliança de mútua proteção contra invasores ou perigos. Biblicamente, podemos notar que apesar dessas garantias hereditárias e alianças serem previstas a um casamento, também há o mandamento divino ao homem e mulher se juntarem e se multiplicarem. Hebreus, judeus e cristãos progridem o entendimento do casamento a ser uma instituição divina, e não apenas um contrato. A reivindicação \emph{queer} não é o reconhecimento de seu casamento como instituição divina parte de um mandamento divino, mas como contrato e instituição civil. Ora, se sua luta tem como objetivo a garantia hereditária e comunhão plena de vida\footnote{Como está descrito no Artigo 1511 do Código Civil: "O casamento estabelece comunhão plena de vida, com base na igualdade de direitos e deveres dos cônjuges."} e não o reconhecimento como instituição divina, não há qualquer ataque a instituições eclesiásticas. Esta separação se torna mais evidente ainda sob a perspectiva protestante de laicidade do Estado.

A união ocidental entre a Igreja Católica e Estados, onde tanto a forma de governo quanto a forma de legislar era permeada pelo catolicismo, impedia a existência de vertentes religiosas paralelas ou opostas e a cidadania de quem as praticava. Por exemplo, a educação leiga era obrigatoriamente católica, os cemitérios eram exclusivos a católicos, e até mesmo o casamento civil era direito apenas de seus fiéis. A luta protestante para que seus direitos de cidadão fossem garantidos necessitou que se desmanchasse a união da religião e do estado, tornando-os assim laicos. É, portanto, incongruente o protestantismo se utilizar de argumentos doutrinários para impedir que cidadãos exerçam sua cidadania de forma plena e igualitária, dado que o entendimento histórico protestante de separação de igreja e estado foi fundamental para sua própria existência. A injustiça e desigualdade foi experimentada pelos protestantes ao ponto de lutarem na esfera cível em prol de si.

Na Constituição Federal do Brasil temos o princípio da igualdade, perante a lei, onde não há nenhuma distinção de qualquer natureza e que garante a inviolabilidade do direito à vida, liberdade, igualdade, segurança e propriedade. Ora, não há como exercer o direito à propriedade de forma plena se não existe a possibilidade de casamento civil entre duas pessoas que constróem patrimônio juntas. A hereditariedade de suas posses se torna comprometida sem as garantias de um casamento. Também não há igualdade sem distinção de qualquer natureza entre cidadãos se dois de sexos diferentes podem se casar e outros dois não podem por serem do mesmo sexo. Ocorrendo a distinção entre o sexo de seu parceiro ou parceira, o princípio norteador de igualdade constitucional é automaticamente ferido. Portanto, toda instituição pública é obrigada a reconhecer o casamento \emph{queer} da mesma forma que reconhece o casamento heterossexual.

Portanto, entendendo os objetivos históricos da existência de casamento em diversas culturas e povos como contratos sociais e patrimoniais, o entendimento protestante de que estados devem permanecer laicos e que o Brasil possui como princípio constitucional de igualdade que todos são iguais perante a lei sem distinção de qualquer natureza, a comunidade \emph{queer} deve ser amparada por toda a sociedade em sua luta para exercer sua cidadania de forma plena e desimpedida. Protestantes podem também participar desta luta, afinal, no passado também foram impedidos de serem vistos pela lei como iguais a católicos.

% \renewcommand{\bibname}{{REFER\^ENCIAS}}
% \bibliography{template_flam.bib}

\end{document}
