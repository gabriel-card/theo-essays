
\documentclass[
	% -- opções da classe memoir --
    article,            % artigo academico
	12pt,				% tamanho da fonte
	%openright,			% capítulos começam em pág ímpar (insere página vazia caso preciso)
	oneside,			% para impressão em recto e verso. Oposto a oneside (twoside)
	a4paper,			% tamanho do papel. 
	% -- opções da classe abntex2 --
	%chapter=TITLE,		% títulos de capítulos convertidos em letras maiúsculas
	%section=TITLE,		% títulos de seções convertidos em letras maiúsculas
	%subsection=TITLE,	% títulos de subseções convertidos em letras maiúsculas
	%subsubsection=TITLE,% títulos de subsubseções convertidos em letras maiúsculas
	% -- opções do pacote babel --
	english,			% idioma adicional para hifenização
	french,				% idioma adicional para hifenização
	spanish,			% idioma adicional para hifenização
	brazil				% o último idioma é o principal do documento
	]{abntex2}

% ---
% Pacotes básicos 
% ---
\usepackage{lmodern}			% Usa a fonte Latin Modern			
\usepackage[T1]{fontenc}		% Selecao de codigos de fonte.
\usepackage[utf8]{inputenc}		% Codificacao do documento (conversão automática dos acentos)
\usepackage{indentfirst}		% Indenta o primeiro parágrafo de cada seção.
\usepackage{color}				% Controle das cores
\usepackage{graphicx}			% Inclusão de gráficos
\usepackage{microtype} 			% para melhorias de justificação
% ---

% ---
% Pacotes de citações
% ---
\usepackage[brazilian,hyperpageref]{backref}	 % Paginas com as citações na bibl
\usepackage[alf]{abntex2cite}	% Citações padrão ABNT

% --- 
% CONFIGURAÇÕES DE PACOTES
% --- 

% ---
% Configurações do pacote backref
% Usado sem a opção hyperpageref de backref
\renewcommand{\backrefpagesname}{Citado na(s) página(s):~}
% Texto padrão antes do número das páginas
\renewcommand{\backref}{}
% Define os textos da citação
\renewcommand*{\backrefalt}[4]{
	\ifcase #1 %
		Nenhuma citação no texto.%
	\or
		Citado na página #2.%
	\else
		Citado #1 vezes nas páginas #2.%
	\fi}%
% ---

% ---
% Informações de dados para CAPA e FOLHA DE ROSTO
% ---
\titulo{HISTÓRIA DA IGREJA: ANTIGA E MEDIEVAL \\ QUESTÃO ABERTA 03}
\autor{GABRIEL CARDOSO DOS SANTOS FALEIRO}
\local{ARUJÁ-SP}
\data{2024}
\instituicao{%
  FLAM - FACULDADE LATINO AMERICANA
}
\tipotrabalho{QUESTÃO ABERTA 03}
% O preambulo deve conter o tipo do trabalho, o objetivo, 
% o nome da instituição e a área de concentração 
\preambulo{Trabalho da disciplina de História da Igreja: Antiga e Medieval, solicitado pelo prof. Paulo Henrique Martins.}
% ---


% ---
% Configurações de aparência do PDF final

% alterando o aspecto da cor azul
\definecolor{blue}{RGB}{41,5,195}

% informações do PDF
\makeatletter
\hypersetup{
     	%pagebackref=true,
		pdftitle={\@title}, 
		pdfauthor={\@author},
    	pdfsubject={\imprimirpreambulo},
	    pdfcreator={LaTeX with abnTeX2},
		pdfkeywords={abnt}{latex}{abntex}{abntex2}{trabalho acadêmico}, 
		colorlinks=true,       		% false: boxed links; true: colored links
    	linkcolor=blue,          	% color of internal links
    	citecolor=blue,        		% color of links to bibliography
    	filecolor=magenta,      		% color of file links
		urlcolor=blue,
		bookmarksdepth=4
}
\makeatother
% ---
% ---
% compila o indice
% ---
\makeindex
% ---

% ----
% Início do documento
% ----
\begin{document}

\citeoption{abnt-full-initials=yes}


% Seleciona o idioma do documento (conforme pacotes do babel)
%\selectlanguage{english}
\selectlanguage{brazil}
% ----------------------------------------------------------
% ELEMENTOS PRÉ-TEXTUAIS
% ----------------------------------------------------------
% \pretextual

% ---
% Capa
% ---
\imprimircapa
% ---

% ---
% Folha de rosto
% (o * indica que haverá a ficha bibliográfica)
% ---
% \imprimirfolhaderosto*
\imprimirfolhaderosto
% ---

% ---
% inserir o sumario
---
\pdfbookmark[0]{\contentsname}{toc}
\tableofcontents*
\cleardoublepage
% ---

% ----------------------------------------------------------
% ELEMENTOS TEXTUAIS
% ----------------------------------------------------------
\textual

% ----------------------------------------------------------
% Introdução (mas presente no Sumário)
% ----------------------------------------------------------

% O Humanismo foi determinante para a criação de um Humanismo cristão e, ainda, para a preparação e formação da maioria dos reformadores cultos ou mestres nos séculos XVI e XVII. Explique como os reformadores foram afetados pelas ideias humanistas de seu tempo.

\section{INTRODUÇÃO}
% ----------------------------------------------------------
Para entendermos como os reformadores foram afetados pelas ideias humanistas de seu tempo, primeiro precisamos pontuar quais ideias surgiram no Humanismo que foram incipientes aos pensamentos dos reformadores. A centralização da humanidade e de sua existência e o desencadeamento das análises das relações da humanidade com o mundo sob essa nova perspectiva, a ideia da dignidade natural da humanidade que traz como consequência seu direito à liberdade de conhecimento entre outras mais formam o \emph{zeitgeist}\footnote{Pode ser entendido como \emph{espírito do tempo}; denota o conjunto da forma de pensar, agir em sociedade e expressões culturais de uma região em uma determindada época da história.} do fim da Idade Média e início da Idade Moderna.

Neste contexto intelectual se encontram alguns alicerces humanistas que permitiram os reformadores idealizarem suas críticas e contrapontos à Igreja Católica Romana. É importante ressaltar que o impacto do Humanismo na Reforma se concentra muito mais na quebra do \emph{status quo} intelectual do que necessariamente na produção teológica reformada. Por George:
\begin{citacao}
O humanismo, assim como o misticismo, foi parte da estrutura que possibilitou aos reformadores questionar certas suposições da tradição recebida, mas que em si mesma não era suficiente para fornecer uma resposta duradoura às obsessivas perguntas da época. \cite[p.49]{GEORGE}
\end{citacao}

\section{O SECULARISMO E A VIDA RELIGIOSA}
O Humanismo trouxe, num resgate próximo dos moldes da Grécia clássica, uma tentativa de secularização intelectual, em contraste com a intelectualidade proposta pela Igreja Católica Romana que trazia a fé como objeto central aos questionamentos filosóficos e não a humanidade. Um movimento natural quando se introduz a humanidade como centro das questões é o resgate das origens dessas mesmas questões dentro da própria história humana; nesta análise os humanistas e posteriormente reformadores se deparam com uma contradição, por Cairns:
\begin{citacao}
Ao comparar a sociedade hierárquica corporativa em que viviam com a liberdade intelectual e o secularismo da sociedade grega, associados ao princípio de liberdade individual propostos pelas Escrituras, as pessoas duvidavam da validade das pretensões da Igreja de Roma e de seus líderes. Começaram, então, a enxergar horizontes intelectuais mais amplos. O interesse das pessoas era agora mais pela vida secular do que pela religiosa. \cite[p. 254]{CAIRNS}
\end{citacao}

\pagebreak

\section{A LIBERDADE AO CONHECIMENTO E O LIVRE EXAME DAS ESCRITURAS}
O entendimento humanista que a humanidade possui naturalmente o direito à liberdade de conhecimento, provinda da centralização da humanidade em questões filosóficas e sociais, contrasta fortemente com o monopólio do conhecimento acerca das Escrituras promovida pela Igreja Católica Romana durante a Idade Média.  A defesa dos reformadores de que todo cristão deve possuir livre acesso às Escrituras e também ter seu livre exame\footnote{Não confundir livre exame com interpretação feita sem critérios, como costuma ser formado o espantalho em discussões vãs.} permite que a congregação julgue, a partir da própria Bíblia, os ensinamentos e pregações expostas por seus ministros.

Este princípio acarreta em uma direta contradição ao monopólio da Igreja Católica Romana sobre as Escrituras, onde era dentro do clero a única possibilidade de acesso e exame das Escrituras. Porém, conforme é recomendado pelos apóstolos, como cristãos temos o dever de examinar e julgar os ensinamentos que nos são dados; como em I João 4 verso primeiro:
\begin{citacao}
Amados, não dei crédito a qualquer espírito; antes, provai os espíritos se procedem de Deus, porque muitos falsos profetas têm saído pelo mundo afora. (1Jo 4.1)
\end{citacao}

\section{CONCLUSÃO}
Entende-se, assim, que o humanismo proporcionou aos reformadores, principalmente, a mudança de forma de pensar necessária para que os questionamentos e propostas de reforma fossem idealizados. Podemos concluir que o humanismo não está presente necessariamente na produção da teologia reformada mas sim na criação das condições que tornaram possíveis e favoráveis ao questionamento inicial que daria o pontapé, assim, nos princípios teológicos que fariam contraponto à hegemonia da época, representada pela Igreja Católica Romana.

\pagebreak
\bibliography{20250419_hi1_qa03.bib}

\end{document}
