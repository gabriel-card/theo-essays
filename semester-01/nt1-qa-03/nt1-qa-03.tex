
\documentclass[
	% -- opções da classe memoir --
    article,            % artigo academico
	12pt,				% tamanho da fonte
	%openright,			% capítulos começam em pág ímpar (insere página vazia caso preciso)
	oneside,			% para impressão em recto e verso. Oposto a oneside (twoside)
	a4paper,			% tamanho do papel. 
	% -- opções da classe abntex2 --
	chapter=TITLE,		% títulos de capítulos convertidos em letras maiúsculas
	section=TITLE,		% títulos de seções convertidos em letras maiúsculas
	%subsection=TITLE,	% títulos de subseções convertidos em letras maiúsculas
	%subsubsection=TITLE,% títulos de subsubseções convertidos em letras maiúsculas
	% -- opções do pacote babel --
	english,			% idioma adicional para hifenização
	french,				% idioma adicional para hifenização
	spanish,			% idioma adicional para hifenização
	brazil				% o último idioma é o principal do documento
	]{abntex2}

% ---
% Pacotes básicos 
% ---
\usepackage{times}				% Usa a fonte Times Roman			
\usepackage[T1]{fontenc}		% Selecao de codigos de fonte.
\usepackage[utf8]{inputenc}		% Codificacao do documento (conversão automática dos acentos)
\usepackage{indentfirst}		% Indenta o primeiro parágrafo de cada seção.
\usepackage{color}				% Controle das cores
\usepackage{graphicx}			% Inclusão de gráficos
\usepackage{microtype} 			% para melhorias de justificação
% ---

% ---
% Pacotes de citações
% ---
\usepackage[brazilian,hyperpageref]{backref}	 % Paginas com as citações na bibl
\usepackage[alf]{abntex2cite}	% Citações padrão ABNT

% --- 
% CONFIGURAÇÕES DE PACOTES
% --- 

% ---
% Configurações do pacote backref
% Usado sem a opção hyperpageref de backref
\renewcommand{\backrefpagesname}{Citado na(s) página(s):~}
% Texto padrão antes do número das páginas
\renewcommand{\backref}{}
% Define os textos da citação
\renewcommand*{\backrefalt}[4]{
	\ifcase #1 %
		Nenhuma citação no texto.%
	\or
		Citado na página #2.%
	\else
		Citado #1 vezes nas páginas #2.%
	\fi}%
% ---
% ---
% FORMATAÇAO FLAM
% ---
\setlength{\parindent}{1.25cm}
\setlength{\parskip}{0.5cm}
\setlength\afterchapskip{\lineskip}
\setlrmarginsandblock{3cm}{2cm}{*}
\setulmarginsandblock{3cm}{2cm}{*}
\checkandfixthelayout
\renewcommand{\ABNTEXchapterfont}{\normalfont}
\renewcommand{\ABNTEXsectionfontsize}{\large\bfseries}
\renewcommand{\cftsectionfont}{\bfseries\MakeTextUppercase}
\renewcommand{\ABNTEXsubsectionfontsize}{\normalsize}
\renewcommand{\cftsubsectionfont}{\normalfont\MakeTextUppercase} % Tirar negrito das subsecoes no sumario
\renewcommand{\ABNTEXsubsubsectionfontsize}{\normalsize\bfseries}
\renewcommand{\cftsubsubsectionfont}{\bfseries} % Tirar negrito das subsecoes no sumario
% ---
% Informações de dados para CAPA e FOLHA DE ROSTO
% ---
\titulo{NOVO TESTAMENTO 1: EVANGELHOS E ATOS \\ QUESTÃO ABERTA 03}
\autor{GABRIEL CARDOSO DOS SANTOS FALEIRO}
\local{ARUJÁ-SP}
\data{2025}
\instituicao{%
  FLAM - FACULDADE LATINO AMERICANA
}
\tipotrabalho{QUESTÃO ABERTA 03}
% O preambulo deve conter o tipo do trabalho, o objetivo, 
% o nome da instituição e a área de concentração 
\preambulo{Trabalho da disciplina de Novo Testamento 1: Evangelhos e Atos, solicitado pelo prof. Dr. Elias Bartolomeu Binja.}
% ---


% ---
% Configurações de aparência do PDF final

% alterando o aspecto da cor azul
\definecolor{blue}{RGB}{41,5,195}

% informações do PDF
\makeatletter
\hypersetup{
     	%pagebackref=true,
		pdftitle={\@title}, 
		pdfauthor={\@author},
    	pdfsubject={\imprimirpreambulo},
	    pdfcreator={Gabriel Cardoso dos Santos Faleiro},
		pdfkeywords={abnt}{latex}{abntex}{abntex2}{trabalho acadêmico}, 
		colorlinks=true,       		% false: boxed links; true: colored links
    	linkcolor=blue,          	% color of internal links
    	citecolor=blue,        		% color of links to bibliography
    	filecolor=magenta,      		% color of file links
		urlcolor=blue,
		bookmarksdepth=4
}
\makeatother
% ---
% ---
% compila o indice
% ---
\makeindex
% ---

% ----
% Início do documento
% ----
\begin{document}

\citeoption{abnt-full-initials=yes}


% Seleciona o idioma do documento (conforme pacotes do babel)
%\selectlanguage{english}
\selectlanguage{brazil}
% ----------------------------------------------------------
% ELEMENTOS PRÉ-TEXTUAIS
% ----------------------------------------------------------
% \pretextual

% ---
% Capa
% ---
\imprimircapa
% ---

% ---
% Folha de rosto
% (o * indica que haverá a ficha bibliográfica)
% ---
% \imprimirfolhaderosto*
\imprimirfolhaderosto
% ---

% ---
% inserir o sumario
% ---
% \pdfbookmark[0]{\contentsname}{toc}
% \tableofcontents*
% \cleardoublepage
% ---

% ----------------------------------------------------------
% ELEMENTOS TEXTUAIS
% ----------------------------------------------------------
\textual
\pagestyle{simple}

% ----------------------------------------------------------
% Introdução (mas presente no Sumário)
% Como os outros três evangelhos sinóticos, o Evangelho de João carrega uma visão de Jesus Cristo pós-pascal, isto é, ele pressupõe que os eventos da sua morte e ressurreição conduziram seus discípulos a uma interpretação bastante diferente da sua existência histórica. Contudo, enquanto os três sinóticos limitam essa interpretação à realidade histórica de Jesus, no Evangelho de João se percebe uma tendência a desligar-se desta para enfatizar inteiramente a interpretação pós-pascal. De fato, João comenta que a vivência histórica de Jesus é impossível de ser entendida no momento do seu acontecimento, e seu significado aguardava por uma revelação posterior mediada pelo Espírito.

% Explique o papel do Espírito na apresentação de Jesus Cristo segundo o Evangelho de João.

\section*{PAPEL DO ESPÍRITO SANTO EM JOÃO}
João apresenta o Espírito Santo como o Parácleto. Esta é uma palavra transliterada do grego (\emph{parakletos}) que dificilmente encontramos uma tradução direta que embarque seu significado de forma plena. Tanto é que, sua transliteração também era comum por judeus na época. Seu sentido mais literal é de "chamado para o lado de", como alguém que é convidado ou pedido a ajudar ou auxiliar em algo ou prestar esse auxílio à alguém. Apesar de algumas traduções entenderem este auxílio como alguém que presta ajuda a defesa de alguém em um tribunal, como um advogado, suas ocorrências no evangelho de João estão mais próximas de um orientador ou ensinador (João 14.16\footnote{Pela ACF: "Mas aquele Consolador, o Espírito Santo, que o Pai enviará em meu nome, esse vos ensinará todas as coisas, e vos fará lembrar de tudo quanto vos tenho dito". Onde \emph{Consolador} foi a palavra escolhida para a tradução de \emph{parakletos}.}) juntamente com a função de testemunhar acerca de Jesus ao mundo (João 15.26\footnote{Pela ACF: "Mas, quando vier o Consolador, que eu da parte do Pai vos hei de enviar, aquele Espírito de verdade, que procede do Pai, ele testificará de mim". Aqui também foi escolhido \emph{Consolador} para traduzir \emph{parakletos}.}). Após a ascenção de Cristo, passamos a ter a presença real e até o fim dos tempos do Parácleto, iniciada no Pentecostes. É, portanto, através das lembranças dos ensinamentos de Cristo e também de seu testemunho que o Espírito Santo cumpre seu papel na Igreja até o fim dos tempos.




% ----------------------------------------------------------

% \renewcommand{\bibname}{{REFER\^ENCIAS}}
% \bibliography{nt1-qa-03.bib}

\end{document}
