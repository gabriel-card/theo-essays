
\documentclass[
	% -- opções da classe memoir --
    article,            % artigo academico
	12pt,				% tamanho da fonte
	%openright,			% capítulos começam em pág ímpar (insere página vazia caso preciso)
	oneside,			% para impressão em recto e verso. Oposto a oneside (twoside)
	a4paper,			% tamanho do papel. 
	% -- opções da classe abntex2 --
	chapter=TITLE,		% títulos de capítulos convertidos em letras maiúsculas
	section=TITLE,		% títulos de seções convertidos em letras maiúsculas
	%subsection=TITLE,	% títulos de subseções convertidos em letras maiúsculas
	%subsubsection=TITLE,% títulos de subsubseções convertidos em letras maiúsculas
	% -- opções do pacote babel --
	english,			% idioma adicional para hifenização
	french,				% idioma adicional para hifenização
	spanish,			% idioma adicional para hifenização
	brazil				% o último idioma é o principal do documento
	]{abntex2}

% ---
% Pacotes básicos 
% ---
\usepackage{times}				% Usa a fonte Times Roman			
\usepackage[T1]{fontenc}		% Selecao de codigos de fonte.
\usepackage[utf8]{inputenc}		% Codificacao do documento (conversão automática dos acentos)
\usepackage{indentfirst}		% Indenta o primeiro parágrafo de cada seção.
\usepackage{color}				% Controle das cores
\usepackage{graphicx}			% Inclusão de gráficos
\usepackage{microtype} 			% para melhorias de justificação
% ---

% ---
% Pacotes de citações
% ---
\usepackage[brazilian,hyperpageref]{backref}	 % Paginas com as citações na bibl
\usepackage[alf]{abntex2cite}	% Citações padrão ABNT

% --- 
% CONFIGURAÇÕES DE PACOTES
% --- 

% ---
% Configurações do pacote backref
% Usado sem a opção hyperpageref de backref
\renewcommand{\backrefpagesname}{Citado na(s) página(s):~}
% Texto padrão antes do número das páginas
\renewcommand{\backref}{}
% Define os textos da citação
\renewcommand*{\backrefalt}[4]{
	\ifcase #1 %
		Nenhuma citação no texto.%
	\or
		Citado na página #2.%
	\else
		Citado #1 vezes nas páginas #2.%
	\fi}%
% ---
% ---
% FORMATAÇAO FLAM
% ---
\setlength{\parindent}{1.25cm}
\setlength{\parskip}{0.5cm}
\setlength\afterchapskip{\lineskip}
\setlrmarginsandblock{3cm}{2cm}{*}
\setulmarginsandblock{3cm}{2cm}{*}
\checkandfixthelayout
\renewcommand{\ABNTEXchapterfont}{\normalfont}
\renewcommand{\ABNTEXsectionfontsize}{\large\bfseries}
\renewcommand{\cftsectionfont}{\bfseries\MakeTextUppercase}
\renewcommand{\ABNTEXsubsectionfontsize}{\normalsize}
\renewcommand{\cftsubsectionfont}{\normalfont\MakeTextUppercase} % Tirar negrito das subsecoes no sumario
\renewcommand{\ABNTEXsubsubsectionfontsize}{\normalsize\bfseries}
\renewcommand{\cftsubsubsectionfont}{\bfseries} % Tirar negrito das subsecoes no sumario
% ---
% Informações de dados para CAPA e FOLHA DE ROSTO
% ---
\titulo{ESPIRITUALIDADE CRISTÃ \\ QUESTÃO ABERTA 02}
\autor{GABRIEL CARDOSO DOS SANTOS FALEIRO}
\local{ARUJÁ-SP}
\data{2025}
\instituicao{%
  FLAM - FACULDADE LATINO AMERICANA
}
\tipotrabalho{QUESTÃO ABERTA 02}
% O preambulo deve conter o tipo do trabalho, o objetivo, 
% o nome da instituição e a área de concentração 
\preambulo{Trabalho da disciplina de Espiritualidade Cristã, solicitado pelo Prof. Dr. André Botelho.}
% ---


% ---
% Configurações de aparência do PDF final

% alterando o aspecto da cor azul
\definecolor{blue}{RGB}{41,5,195}

% informações do PDF
\makeatletter
\hypersetup{
     	%pagebackref=true,
		pdftitle={\@title}, 
		pdfauthor={\@author},
    	pdfsubject={\imprimirpreambulo},
	    pdfcreator={Gabriel Cardoso dos Santos Faleiro},
		pdfkeywords={abnt}{latex}{abntex}{abntex2}{trabalho acadêmico}, 
		colorlinks=true,       		% false: boxed links; true: colored links
    	linkcolor=blue,          	% color of internal links
    	citecolor=blue,        		% color of links to bibliography
    	filecolor=magenta,      		% color of file links
		urlcolor=blue,
		bookmarksdepth=4
}
\makeatother
% ---
% ---
% compila o indice
% ---
\makeindex
% ---

% ----
% Início do documento
% ----
\begin{document}

\citeoption{abnt-full-initials=yes}


% Seleciona o idioma do documento (conforme pacotes do babel)
%\selectlanguage{english}
\selectlanguage{brazil}
% ----------------------------------------------------------
% ELEMENTOS PRÉ-TEXTUAIS
% ----------------------------------------------------------
% \pretextual

% ---
% Capa
% ---
\imprimircapa
% ---

% ---
% Folha de rosto
% (o * indica que haverá a ficha bibliográfica)
% ---
% \imprimirfolhaderosto*
\imprimirfolhaderosto
% ---

% ---
% inserir o sumario
% ---
% \pdfbookmark[0]{\contentsname}{toc}
% \tableofcontents*
% \cleardoublepage
% ---

% ----------------------------------------------------------
% ELEMENTOS TEXTUAIS
% ----------------------------------------------------------
\textual
\pagestyle{simple}

% ----------------------------------------------------------
% Introdução (mas presente no Sumário)

% Descreva quais foram na sua opinião as manifestações de virtude e também de contradição na espiritualidade evangélica brasileira durante este período de pandemia e quarentena que estamos vivendo nestes dias.


\section*{A PANDEMIA E A ESPIRITUALIDADE EVANGÉLICA BRASILEIRA}
% ----------------------------------------------------------
Entendendo a abrangência dos termos espiritualidade evangélica brasileira como toda a práxis e mentalidade observada nas expressões comunitárias de fé dos evangélicos do Brasil, podemos notar contradições e virtudes em suas atitudes durante a pandemia global e quarentena. Diante de um cenário onde aglomerações não são possíveis por motivos sanitários, a igreja evangélica se deparou com alguns dilemas, entre eles: como cultuar a Deus de forma comunitária enquanto separados fisicamente?

A impossibilidade de congregar trouxe o desafio das igrejas de se reinventarem para manter sua vida comunitária em andamento. Com a amplitude de alcance da internet e facilidade de transmissões ao vivo, igrejas a partir do trabalho voluntários de seus membros, conseguiram se tornar presentes no mundo virtual. Apesar de ter sido uma resposta a um momento de quarentena e direcionado a seus próprios membros, criou-se um cenário onde a Palavra é pregada para uma amplitude imensa de pessoas que, outrora, jamais teriam tido contato com aquela comunidade. Manifesta-se, assim, uma forma sem precedentes na história de culto a Deus: cada família em sua própria casa prestando o mesmo culto, ao mesmo tempo, em comunhão, mesmo distantes fisicamente.

Porém isto não foi universal: existiram igrejas resistentes a quarentena que se recusaram a fechar suas portas e continuaram com os cultos presenciais de forma ordinária. Estas igrejas, e aqui dispensando o comentário político-partidário dada a complexidade do tema que este texto não se propõe, entendiam que o direito de prestar culto era primário ao direito sanitário da sociedade como um todo. Sobrepuseram a livre manifestação de culto prevista em lei em detrimento da saúde pública, colocando seus próprios fiéis em risco e toda a sociedade ao permitir aglomerações durante uma janela de tempo extremamente sensível a transmissão de um vírus que, hoje, ceifou aproximadamente 716 mil vidas\footnote{Informação retirada do site Coronavirus Brasil, construído pela \citeonline{CoronavirusBrasil}.}. Não há como mensurar o impacto direto dessas igrejas que se recusaram a fechar as portas com o número de mortes, mas é plausível dizer que esta atitude, contrária as normas sanitárias, não ajudaram a diminuir as vidas ceifadas. Ora, se uma igreja tem como mais importante não cessar seus cultos presenciais mesmo que isso signifique a morte de pessoas, questiona-se se a motivação dos cultos é, de fato, a adoração a Deus de forma comunitária. Afinal, o sacrifício que muitas igrejas fizeram de fechar suas portas foi um ato de amor e cuidado com a própria comunidade e sociedade em que está inserida.

Posta esta contradição, podemos perceber como as manifestações da espiritualidade evangélica brasileira foram tanto virtuosas como contraditórias. Apesar dos evangélicos serem vistos como uma única camada social, é um povo diverso e com formas de expressão de espiritualidade também diversas. É imperativo que, ao exercemos nossa espiritualidade, nós nos atentemos ao nosso redor: será que realmente existe uma conspiração para nos impedir de prestar culto a Deus, ou será que isso é fabricação de líderes para manutenção do status quo? Será que nós deveriamos reivindicar direitos que, em seu exercício, afetará de forma mortal nosso próximo? Que tenhamos, como Igreja, a postura esperada por Cristo: 
\begin{citacao}
"O meu mandamento é este: Que vos ameis uns aos outros, assim como eu vos amei. Ninguém tem maior amor do que este, de dar alguém a sua vida pelos seus amigos". (João 15.12-13)	
\end{citacao}


\pagebreak
\renewcommand{\bibname}{{REFER\^ENCIAS}}
\bibliography{ec1-qa-02.bib}

\end{document}
