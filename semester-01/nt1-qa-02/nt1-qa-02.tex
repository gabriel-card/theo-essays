
\documentclass[
	% -- opções da classe memoir --
    article,            % artigo academico
	12pt,				% tamanho da fonte
	%openright,			% capítulos começam em pág ímpar (insere página vazia caso preciso)
	oneside,			% para impressão em recto e verso. Oposto a oneside (twoside)
	a4paper,			% tamanho do papel. 
	% -- opções da classe abntex2 --
	chapter=TITLE,		% títulos de capítulos convertidos em letras maiúsculas
	section=TITLE,		% títulos de seções convertidos em letras maiúsculas
	%subsection=TITLE,	% títulos de subseções convertidos em letras maiúsculas
	%subsubsection=TITLE,% títulos de subsubseções convertidos em letras maiúsculas
	% -- opções do pacote babel --
	english,			% idioma adicional para hifenização
	french,				% idioma adicional para hifenização
	spanish,			% idioma adicional para hifenização
	brazil				% o último idioma é o principal do documento
	]{abntex2}

% ---
% Pacotes básicos 
% ---
\usepackage{times}				% Usa a fonte Times Roman			
\usepackage[T1]{fontenc}		% Selecao de codigos de fonte.
\usepackage[utf8]{inputenc}		% Codificacao do documento (conversão automática dos acentos)
\usepackage{indentfirst}		% Indenta o primeiro parágrafo de cada seção.
\usepackage{color}				% Controle das cores
\usepackage{graphicx}			% Inclusão de gráficos
\usepackage{microtype} 			% para melhorias de justificação
% ---

% ---
% Pacotes de citações
% ---
\usepackage[brazilian,hyperpageref]{backref}	 % Paginas com as citações na bibl
\usepackage[alf]{abntex2cite}	% Citações padrão ABNT

% --- 
% CONFIGURAÇÕES DE PACOTES
% --- 

% ---
% Configurações do pacote backref
% Usado sem a opção hyperpageref de backref
\renewcommand{\backrefpagesname}{Citado na(s) página(s):~}
% Texto padrão antes do número das páginas
\renewcommand{\backref}{}
% Define os textos da citação
\renewcommand*{\backrefalt}[4]{
	\ifcase #1 %
		Nenhuma citação no texto.%
	\or
		Citado na página #2.%
	\else
		Citado #1 vezes nas páginas #2.%
	\fi}%
% ---
% ---
% FORMATAÇAO FLAM
% ---
\setlength{\parindent}{1.25cm}
\setlength{\parskip}{0.5cm}
\setlength\afterchapskip{\lineskip}
\setlrmarginsandblock{3cm}{2cm}{*}
\setulmarginsandblock{3cm}{2cm}{*}
\checkandfixthelayout
\renewcommand{\ABNTEXchapterfont}{\normalfont}
\renewcommand{\ABNTEXsectionfontsize}{\large\bfseries}
\renewcommand{\cftsectionfont}{\bfseries\MakeTextUppercase}
\renewcommand{\ABNTEXsubsectionfontsize}{\normalsize}
\renewcommand{\cftsubsectionfont}{\normalfont\MakeTextUppercase} % Tirar negrito das subsecoes no sumario
\renewcommand{\ABNTEXsubsubsectionfontsize}{\normalsize\bfseries}
\renewcommand{\cftsubsubsectionfont}{\bfseries} % Tirar negrito das subsecoes no sumario
% ---
% Informações de dados para CAPA e FOLHA DE ROSTO
% ---
\titulo{NOVO TESTAMENTO 1: EVANGELHO E ATOS \\ QUESTÃO ABERTA 02}
\autor{GABRIEL CARDOSO DOS SANTOS FALEIRO}
\local{ARUJÁ-SP}
\data{2025}
\instituicao{%
  FLAM - FACULDADE LATINO AMERICANA
}
\tipotrabalho{QUESTÃO ABERTA 02}
% O preambulo deve conter o tipo do trabalho, o objetivo, 
% o nome da instituição e a área de concentração 
\preambulo{Trabalho da disciplina de Novo Testamento 1: Evangelho e Atos, solicitado pelo prof. Dr. Elias Bartolomeu Binja.}
% ---


% ---
% Configurações de aparência do PDF final

% alterando o aspecto da cor azul
\definecolor{blue}{RGB}{41,5,195}

% informações do PDF
\makeatletter
\hypersetup{
     	%pagebackref=true,
		pdftitle={\@title}, 
		pdfauthor={\@author},
    	pdfsubject={\imprimirpreambulo},
	    pdfcreator={Gabriel Cardoso dos Santos Faleiro},
		pdfkeywords={abnt}{latex}{abntex}{abntex2}{trabalho acadêmico}, 
		colorlinks=true,       		% false: boxed links; true: colored links
    	linkcolor=blue,          	% color of internal links
    	citecolor=blue,        		% color of links to bibliography
    	filecolor=magenta,      		% color of file links
		urlcolor=blue,
		bookmarksdepth=4
}
\makeatother
% ---
% ---
% compila o indice
% ---
\makeindex
% ---

% ----
% Início do documento
% ----
\begin{document}

\citeoption{abnt-full-initials=yes}


% Seleciona o idioma do documento (conforme pacotes do babel)
%\selectlanguage{english}
\selectlanguage{brazil}
% ----------------------------------------------------------
% ELEMENTOS PRÉ-TEXTUAIS
% ----------------------------------------------------------
% \pretextual

% ---
% Capa
% ---
\imprimircapa
% ---

% ---
% Folha de rosto
% (o * indica que haverá a ficha bibliográfica)
% ---
% \imprimirfolhaderosto*
\imprimirfolhaderosto
% ---

% ---
% inserir o sumario
% ---
% \pdfbookmark[0]{\contentsname}{toc}
% \tableofcontents*
% \cleardoublepage
% ---

% ----------------------------------------------------------
% ELEMENTOS TEXTUAIS
% ----------------------------------------------------------
\textual
\pagestyle{simple}

% ----------------------------------------------------------
% “Conclui-se que todo esse movimento possui uma função no Quarto Evangelho, pois progressivamente Tenda/Santuário e Templo são elevados ao seu ápice quando o novo Templo será o Pai e o Filho: “Ele, porém, falava do templo do seu corpo” (Jo 2,21); “Não vi nenhum templo nela, pois o seu templo é o Senhor, o Deus todo-poderoso, e o Cordeiro” (Ap 21,22). E o sentido próprio desse movimento encontra-se na narrativa da “Mulher Samaritana”, na qual ela indaga sobre o verdadeiro santuário no qual se deve adorar à Deus, Garizim ou Jerusalém, seguida da resposta de Jesus, que não será nem em um e nem no outro, mas em Espírito e Verdade (cf. Jo 20,24).” (ARAUJO, Gilvan Leite de. Do santuário do deserto ao santuário do corpo de Jesus. In: Revista Caminhando, v. 24, n. 2, p. 139, 2019).

% A partir do texto comente sobre a relação feita por Jesus entre o Templo de Jerusalém e o seu corpo como o novo lugar onde seria feita a adoração a Deus, e explique a importância dessa compreensão para o culto cristão nos dias de hoje. 

\section*{}

A união das tradições do norte e sul na narrativa joanina se encontrando em Jesus como o novo e eterno Templo, encerrando assim a disputa do verdadeiro santuário narrado no encontro da "Mulher Samaritana", direciona Jesus como o cumprimento das profecias do segundo capítulo de Isaías e do segundo capítulo de Ageu: onde "a glória desta última casa será maior do que a da primeira, diz o Senhor dos Exércitos, e neste lugar darei a paz, diz o Senhor dos Exércitos." (Ag 2.9)\footnote{AGEU. In: A BÍBLIA SAGRADA: Almeida Corrigida Fiel. São Paulo, 2011.} e "[...] acontecerá nos últimos dias que se firmará o monte da casa do Senhor no cume dos montes, e se elevará por cima dos outeiros; e concorrerão a ele todas as nações." (Is 2.2)\footnote{ISAÍAS. In: A BÍBLIA SAGRADA: Almeida Corrigida Fiel. São Paulo, 2011.}. Sendo Cristo o novo e último Templo, é nEle e através dEle que todas as nações agora possuem acesso aos objetivos da existência dos templos anteriores: a realização do sacrifício para expiação de pecados, representado pela morte de Jesus como Cordeiro de Deus; o pastoreio na Palavra revelada e a adoração ao Deus vivo.

Algumas implicações se tornam pertinentes diante desta mudança de paradigma. Ora, sendo Jesus o novo Templo e habitando em nós o Espírito Santo, somos também de algum modo o templo de Deus? Paulo em sua primeira carta aos Coríntios parece concordar com esta implicação lógica: "Não sabeis vós que sois o templo de Deus e que o Espírito de Deus habita em vós? Se alguém destruir o templo de Deus, Deus o destruirá; porque o templo de Deus, que sois vós, é santo." (1Co 3.16-17)\footnote{1 CORÍNTIOS. In: A BÍBLIA SAGRADA: Almeida Corrigida Fiel. São Paulo, 2011.}

Sendo nós, cristãos, templos de Deus por habitar em nós o Espírito Santo que é, também, a presença real de Jesus, conclui-se que ao nos reunirmos em culto também trazemos conosco as mesmas responsabilidades que outrora eram do templo: o sacrifício, o pastoreio e a adoração. Em outros termos, o culto cristão deve ter como objetivo a pregação do Evangelho, apresentando Cristo como o Cordeiro de Deus que retira o pecado do mundo, tornando possível a comunhão de todos povos com Deus, guiando-os em adoração.




\pagebreak
% \renewcommand{\bibname}{{REFER\^ENCIAS}}
% \bibliography{nt1-qa-02.bib}

\end{document}
