
\documentclass[
	% -- opções da classe memoir --
    article,            % artigo academico
	12pt,				% tamanho da fonte
	%openright,			% capítulos começam em pág ímpar (insere página vazia caso preciso)
	oneside,			% para impressão em recto e verso. Oposto a oneside (twoside)
	a4paper,			% tamanho do papel. 
	% -- opções da classe abntex2 --
	chapter=TITLE,		% títulos de capítulos convertidos em letras maiúsculas
	section=TITLE,		% títulos de seções convertidos em letras maiúsculas
	subsection=TITLE,	% títulos de subseções convertidos em letras maiúsculas
	%subsubsection=TITLE,% títulos de subsubseções convertidos em letras maiúsculas
	% -- opções do pacote babel --
	english,			% idioma adicional para hifenização
	french,				% idioma adicional para hifenização
	spanish,			% idioma adicional para hifenização
	brazil				% o último idioma é o principal do documento
	]{abntex2}

% ---
% Pacotes básicos 
% ---
\usepackage{times}				% Usa a fonte Times Roman	
\usepackage[T1]{fontenc}		% Selecao de codigos de fonte.
\usepackage[utf8]{inputenc}		% Codificacao do documento (conversão automática dos acentos)
\usepackage{indentfirst}		% Indenta o primeiro parágrafo de cada seção.
\usepackage{color}				% Controle das cores
\usepackage{graphicx}			% Inclusão de gráficos
\usepackage{microtype} 			% para melhorias de justificação
% ---

% ---
% Pacotes de citações
% ---
\usepackage[brazilian,hyperpageref]{backref}	 % Paginas com as citações na bibl
\usepackage[alf]{abntex2cite}	% Citações padrão ABNT

% --- 
% CONFIGURAÇÕES DE PACOTES
% --- 

% ---
% Configurações do pacote backref
% Usado sem a opção hyperpageref de backref
\renewcommand{\backrefpagesname}{Citado na(s) página(s):~}
% Texto padrão antes do número das páginas
\renewcommand{\backref}{}
% Define os textos da citação
\renewcommand*{\backrefalt}[4]{
	\ifcase #1 %
		Nenhuma citação no texto.%
	\or
		Citado na página #2.%
	\else
		Citado #1 vezes nas páginas #2.%
	\fi}%
% ---

% ---
% FORMATAÇAO FLAM
% ---
\setlength{\parindent}{1.25cm}
\setlength{\parskip}{0.5cm}
\setlength\afterchapskip{\lineskip}
\setlrmarginsandblock{3cm}{2cm}{*}
\setulmarginsandblock{3cm}{2cm}{*}
\checkandfixthelayout
\renewcommand{\ABNTEXsectionfontsize}{\large\bfseries}
\renewcommand{\cftsectionfont}{\bfseries\MakeTextUppercase}
\renewcommand{\ABNTEXsubsectionfontsize}{\normalsize}
\renewcommand{\cftsubsectionfont}{\normalfont\MakeTextUppercase} % Tirar negrito das subsecoes no sumario
\renewcommand{\ABNTEXsubsubsectionfontsize}{\normalsize\bfseries}
\renewcommand{\cftsubsubsectionfont}{\bfseries} % Tirar negrito das subsecoes no sumario
% ---
% Informações de dados para CAPA e FOLHA DE ROSTO
% ---
\titulo{HISTÓRIA DA IGREJA: ANTIGA E MEDIEVAL \\ QUESTÃO ABERTA 02}
\autor{GABRIEL CARDOSO DOS SANTOS FALEIRO}
\local{ARUJÁ-SP}
\data{2024}
\instituicao{%
  FLAM - FACULDADE LATINO AMERICANA
}
\tipotrabalho{QUESTÃO ABERTA 02}
% O preambulo deve conter o tipo do trabalho, o objetivo, 
% o nome da instituição e a área de concentração 
\preambulo{Trabalho da disciplina de História da Igreja: Antiga e Medieval, solicitado pelo prof. Paulo Henrique Martins.}
% ---


% ---
% Configurações de aparência do PDF final

% alterando o aspecto da cor azul
\definecolor{blue}{RGB}{41,5,195}

% informações do PDF
\makeatletter
\hypersetup{
     	%pagebackref=true,
		pdftitle={\@title}, 
		pdfauthor={\@author},
    	pdfsubject={\imprimirpreambulo},
	    pdfcreator={LaTeX with abnTeX2},
		pdfkeywords={abnt}{latex}{abntex}{abntex2}{trabalho acadêmico}, 
		colorlinks=true,       		% false: boxed links; true: colored links
    	linkcolor=blue,          	% color of internal links
    	citecolor=blue,        		% color of links to bibliography
    	filecolor=magenta,      		% color of file links
		urlcolor=blue,
		bookmarksdepth=4
}
\makeatother
% ---
% ---
% compila o indice
% ---
\makeindex
% ---

% ----
% Início do documento
% ----
\begin{document}

\citeoption{abnt-full-initials=yes}


% Seleciona o idioma do documento (conforme pacotes do babel)
%\selectlanguage{english}
\selectlanguage{brazil}
% ----------------------------------------------------------
% ELEMENTOS PRÉ-TEXTUAIS
% ----------------------------------------------------------
% \pretextual

% ---
% Capa
% ---
\imprimircapa
% ---

% ---
% Folha de rosto
% (o * indica que haverá a ficha bibliográfica)
% ---
% \imprimirfolhaderosto*
\imprimirfolhaderosto
% ---

% ---
% inserir o sumario
---
\pdfbookmark[0]{\contentsname}{toc}
\tableofcontents*
\cleardoublepage
% ---

% ----------------------------------------------------------
% ELEMENTOS TEXTUAIS
% ----------------------------------------------------------
\textual
\pagestyle{simple}

% ----------------------------------------------------------
% Introdução (mas presente no Sumário)
% ----------------------------------------------------------

\section{Introdução}
% ----------------------------------------------------------
O Humanismo e o renascimento foram movimentos filosóficos e culturais situados entre o final da Idade Média e o início da Modernidade. Ambos marcaram uma transição significativa na forma de pensar e compreender o mundo ao oferecerem um contraponto necessário ao dogmatismo presente na institucionalização da Igreja Católica Romana, que por sua vez buscava se consolidar como religião dominante, trazendo todo o mundo conhecido para a órbita da sua confissão de fé em um Deus único e em seu Filho Jesus Cristo. Humanismo e Renascimento representam, portanto, o início da transição para um pensamento crítico, científico e centrado na natureza humana.

\section{Características do Humanismo}
O Humanismo surge em círculos eruditos europeus que buscavam a emancipação do pensamento, até então sob controle da Igreja Católica Romana. É marcado pela valorização da existência humana, de suas expectativas e necessidades. O homem passa a ser o tema central das discussões, compreendido como um indivíduo capaz de ser livre, adquirir conhecimento e buscar satisfação pessoal.

O Humanismo inaugura um movimento contracultural que promove uma revolução significativa na emancipação do homem europeu. Rompe com a tradição medieval e influencia diretamente a criação do que se pode chamar de humanismo cristão, exercendo forte impacto sobre reformadores dos séculos XVI e XVII, como Calvino e Lutero. Nesse sentido, o Humanismo levanta questionamentos que serão respondidos de forma pragmática pelos movimentos posteriores: o renascimento e as reformas protestantes.

\section{Características do Renascimento}
O Renascimento é um período de transição que marca a passagem para a modernidade ocidental. Nesse momento, elementos anteriormente centrais — promovidos pelo catolicismo durante a Idade Média — tornam-se marginais frente às novas discussões em ebulição.

O homem passa a ser o elemento central, educando-se por meio de reflexões profundas sobre a natureza e a sociedade que o rodeia independente do monopólio da transcendentalidade oferecida pela Igreja Católica Romana.

Os renascentistas não produzem uma ruptura abrupta com as gerações anteriores, mas sim uma diferenciação cultural gradual, que rompe com o dogmatismo ao longo de aproximadamente três séculos.

Durante esse período, as estruturas rígidas e inflexíveis da Idade Média são substituídas por questionamentos, dúvidas e pela redescoberta do homem e de sua obra como reflexo de si mesmo e de suas interações com o mundo.

\section{Conclusão}
O Humanismo e o Renascimento foram fundamentais para o surgimento de uma nova forma de pensar, na qual o ser humano passou a ser valorizado em sua individualidade, liberdade e capacidade racional. Esses movimentos prepararam o terreno para transformações culturais, religiosas e científicas profundas, marcando o fim do domínio absoluto da tradição medieval e abrindo caminho para a modernidade ocidental.


% \pagebreak
% \bibliography{20250226_hi1_qa01.bib}
\end{document}
