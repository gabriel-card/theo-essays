
\documentclass[
	% -- opções da classe memoir --
    article,            % artigo academico
	12pt,				% tamanho da fonte
	%openright,			% capítulos começam em pág ímpar (insere página vazia caso preciso)
	oneside,			% para impressão em recto e verso. Oposto a oneside (twoside)
	a4paper,			% tamanho do papel. 
	% -- opções da classe abntex2 --
	%chapter=TITLE,		% títulos de capítulos convertidos em letras maiúsculas
	%section=TITLE,		% títulos de seções convertidos em letras maiúsculas
	%subsection=TITLE,	% títulos de subseções convertidos em letras maiúsculas
	%subsubsection=TITLE,% títulos de subsubseções convertidos em letras maiúsculas
	% -- opções do pacote babel --
	english,			% idioma adicional para hifenização
	french,				% idioma adicional para hifenização
	spanish,			% idioma adicional para hifenização
	brazil				% o último idioma é o principal do documento
	]{abntex2}

% ---
% Pacotes básicos 
% ---
\usepackage{lmodern}			% Usa a fonte Latin Modern			
\usepackage[T1]{fontenc}		% Selecao de codigos de fonte.
\usepackage[utf8]{inputenc}		% Codificacao do documento (conversão automática dos acentos)
\usepackage{indentfirst}		% Indenta o primeiro parágrafo de cada seção.
\usepackage{color}				% Controle das cores
\usepackage{graphicx}			% Inclusão de gráficos
\usepackage{microtype} 			% para melhorias de justificação
% ---

% ---
% Pacotes de citações
% ---
\usepackage[brazilian,hyperpageref]{backref}	 % Paginas com as citações na bibl
\usepackage[alf]{abntex2cite}	% Citações padrão ABNT

% --- 
% CONFIGURAÇÕES DE PACOTES
% --- 

% ---
% Configurações do pacote backref
% Usado sem a opção hyperpageref de backref
\renewcommand{\backrefpagesname}{Citado na(s) página(s):~}
% Texto padrão antes do número das páginas
\renewcommand{\backref}{}
% Define os textos da citação
\renewcommand*{\backrefalt}[4]{
	\ifcase #1 %
		Nenhuma citação no texto.%
	\or
		Citado na página #2.%
	\else
		Citado #1 vezes nas páginas #2.%
	\fi}%
% ---

% ---
% Informações de dados para CAPA e FOLHA DE ROSTO
% ---
\titulo{HISTÓRIA DA IGREJA: ANTIGA E MEDIEVAL \\ QUESTÃO ABERTA 01}
\autor{GABRIEL CARDOSO DOS SANTOS FALEIRO}
\local{ARUJÁ-SP}
\data{2024}
\instituicao{%
  FLAM - FACULDADE LATINO AMERICANA
}
\tipotrabalho{QUESTÃO ABERTA 01}
% O preambulo deve conter o tipo do trabalho, o objetivo, 
% o nome da instituição e a área de concentração 
\preambulo{Trabalho da disciplina de História da Igreja: Antiga e Medieval, solicitado pelo prof. Paulo Henrique Martins.}
% ---


% ---
% Configurações de aparência do PDF final

% alterando o aspecto da cor azul
\definecolor{blue}{RGB}{41,5,195}

% informações do PDF
\makeatletter
\hypersetup{
     	%pagebackref=true,
		pdftitle={\@title}, 
		pdfauthor={\@author},
    	pdfsubject={\imprimirpreambulo},
	    pdfcreator={LaTeX with abnTeX2},
		pdfkeywords={abnt}{latex}{abntex}{abntex2}{trabalho acadêmico}, 
		colorlinks=true,       		% false: boxed links; true: colored links
    	linkcolor=blue,          	% color of internal links
    	citecolor=blue,        		% color of links to bibliography
    	filecolor=magenta,      		% color of file links
		urlcolor=blue,
		bookmarksdepth=4
}
\makeatother
% ---
% ---
% compila o indice
% ---
\makeindex
% ---

% ----
% Início do documento
% ----
\begin{document}

\citeoption{abnt-full-initials=yes}


% Seleciona o idioma do documento (conforme pacotes do babel)
%\selectlanguage{english}
\selectlanguage{brazil}
% ----------------------------------------------------------
% ELEMENTOS PRÉ-TEXTUAIS
% ----------------------------------------------------------
% \pretextual

% ---
% Capa
% ---
\imprimircapa
% ---

% ---
% Folha de rosto
% (o * indica que haverá a ficha bibliográfica)
% ---
% \imprimirfolhaderosto*
\imprimirfolhaderosto
% ---

% ---
% inserir o sumario
---
\pdfbookmark[0]{\contentsname}{toc}
\tableofcontents*
\cleardoublepage
% ---

% ----------------------------------------------------------
% ELEMENTOS TEXTUAIS
% ----------------------------------------------------------
\textual

% ----------------------------------------------------------
% Introdução (mas presente no Sumário)
% ----------------------------------------------------------
\section{INTRODUÇÃO}
% ----------------------------------------------------------
Este trabalho pretende descrever, de forma breve, diferenças entre memória e história e destacando no fim seu uso para historiadores da Igreja.

\section{MEMÓRIA}
Pode-se definir memória, de forma sucinta e pragmática, como a capacidade pertencente a algo ou alguém de adquirir e armazenar informações. Contextualizando memória na experiência humana, temos uma expansão dessa definição; não apenas é a capacidade de armazenamento de informações mas também o recurso disponível ao indivíduo de entrar novamente em contato com o passado vivido. Esse processo de reencontro com o passado vivido, também chamado de lembrança, se dá de forma orgânica e passível de diversos agentes, conscientes ou inconscientes, que podem transformar a informação armazenada em algo diferente. Esses agentes, como o estado emocional do indíviduo no momento, traumas, entre outros, são os responsáveis pelo valor subjetivo da memória.

Entendendo que a experiência humana é, também, uma experiência social, criamos então um cenário onde memórias individuais passam a ser compartilhadas entre aqueles indivíduos que compartilham de atributos comuns, como língua, território, cultura; que compartilham uma comunidade. Experiências humanas sociais, portanto, não apenas se tornam uma memória de um indivíduo mas também de todos aqueles envolvidos nessas mesmas experiências, criando assim um repositório de informações sobre aquela comunidade.

Apesar de grande subjetividade ser embarcada na memória de apenas um indivíduo, comparado com uma comunidade inteira relembrando uma memória compartilhada entre si, cria-se um cenário onde essas subjetividades individuais se abafam e se ergue uma lembrança comum; construindo assim uma memória coletiva. Obviamente ainda existem subjetividades relacionadas a memórias coletivas, por exemplo uma nação conquistada tem uma memória bem diferente da nação que a conquistou, e vice-versa; mas seu valor ainda é altíssimo, pois são essas memórias que podem definir e dar identidade a aquela nação ou comunidade.

\section{HISTÓRIA}
A história, por sua vez, não pode ser exclusivamente definida como uma mera reconstrução da narração dos fatos passados. Este entendimento leva ao descarte direto das memórias em uma busca infrutífera por documentos ou artigos arqueológicos completamente isentos de qualquer subjetividade. Infrutífera porque todo documento ou artigo arqueológico foi primeiramente criado por uma pessoa pertencente a uma comunidade dentro do espaço-tempo, com todas suas subjetividades individuais e coletivas.

A palavra história é derivada do verbo grego \emph{historeo} que foi muito utilizado originalmente com o significado de aprender pela pesquisa ou investigação. Tomando emprestado esse significado, se faz necessário entender história como também o processo de pesquisa e investigação crítica acerca das memórias coletivas pertencentes a uma comunidade. Transformando assim essas memórias em objeto da história e, por fim, resultando dessa pesquisa o conhecimento dos fatos passados que um dia foram vividos e experienciados por uma comunidade, uma nação ou um povo.

\section{CONCLUSÃO}
Entendendo portanto a história como o processo de produção de conhecimento de fatos vividos e passados a partir --- também --- da pesquisa e investigação das memórias coletivas pertencentes a um povo, e possuindo documentos onde as memórias da comunidade originária da Igreja criados por seus próprios integrantes, cria-se um cenário propício para o resgate e reconstrução da história da Igreja por seus historiadores.



% \pagebreak
% \bibliography{20250226_hi1_qa01.bib}

\end{document}
