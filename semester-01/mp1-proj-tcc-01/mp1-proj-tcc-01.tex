
\documentclass[
	% -- opções da classe memoir --
    article,            % artigo academico
	12pt,				% tamanho da fonte
	%openright,			% capítulos começam em pág ímpar (insere página vazia caso preciso)
	oneside,			% para impressão em recto e verso. Oposto a oneside (twoside)
	a4paper,			% tamanho do papel. 
	% -- opções da classe abntex2 --
	chapter=TITLE,		% títulos de capítulos convertidos em letras maiúsculas
	section=TITLE,		% títulos de seções convertidos em letras maiúsculas
	%subsection=TITLE,	% títulos de subseções convertidos em letras maiúsculas
	%subsubsection=TITLE,% títulos de subsubseções convertidos em letras maiúsculas
	% -- opções do pacote babel --
	english,			% idioma adicional para hifenização
	french,				% idioma adicional para hifenização
	spanish,			% idioma adicional para hifenização
	brazil				% o último idioma é o principal do documento
	]{abntex2}

% ---
% Pacotes básicos 
% ---
\usepackage{times}				% Usa a fonte Times Roman			
\usepackage[T1]{fontenc}		% Selecao de codigos de fonte.
\usepackage[utf8]{inputenc}		% Codificacao do documento (conversão automática dos acentos)
\usepackage{indentfirst}		% Indenta o primeiro parágrafo de cada seção.
\usepackage{color}				% Controle das cores
\usepackage{graphicx}			% Inclusão de gráficos
\usepackage{microtype} 			% para melhorias de justificação
% ---

% ---
% Pacotes de citações
% ---
\usepackage[brazilian,hyperpageref]{backref}	 % Paginas com as citações na bibl
\usepackage[alf]{abntex2cite}	% Citações padrão ABNT

% --- 
% CONFIGURAÇÕES DE PACOTES
% --- 

% ---
% Configurações do pacote backref
% Usado sem a opção hyperpageref de backref
\renewcommand{\backrefpagesname}{Citado na(s) página(s):~}
% Texto padrão antes do número das páginas
\renewcommand{\backref}{}
% Define os textos da citação
\renewcommand*{\backrefalt}[4]{
	\ifcase #1 %
		%
	\or
		Citado na página #2.%
	\else
		Citado #1 vezes nas páginas #2.%
	\fi}%
% ---
% ---
% FORMATAÇAO FLAM
% ---
\setlength{\parindent}{1.25cm}
\setlength{\parskip}{0.5cm}
\setlength\afterchapskip{\lineskip}
\setlrmarginsandblock{3cm}{2cm}{*}
\setulmarginsandblock{3cm}{2cm}{*}
\checkandfixthelayout
\renewcommand{\ABNTEXchapterfont}{\normalfont}
\renewcommand{\ABNTEXsectionfontsize}{\large\bfseries}
\renewcommand{\cftsectionfont}{\bfseries\MakeTextUppercase}
\renewcommand{\ABNTEXsubsectionfontsize}{\normalsize}
\renewcommand{\cftsubsectionfont}{\normalfont\MakeTextUppercase} % Tirar negrito das subsecoes no sumario
\renewcommand{\ABNTEXsubsubsectionfontsize}{\normalsize\bfseries}
\renewcommand{\cftsubsubsectionfont}{\bfseries} % Tirar negrito das subsecoes no sumario
% ---
% Informações de dados para CAPA e FOLHA DE ROSTO
% ---
\titulo{PROJETO TCC \\ A FORMAÇÃO DE COMUNIDADES CRISTÃS SOB UMA PERSPECTIVA ESCATOLÓGICA: CONSEQUÊNCIAS DA EXPECTATIVA ESCATOLÓGICA NAS ORIGENS DE UMA IGREJA}
\autor{GABRIEL CARDOSO DOS SANTOS FALEIRO}
\local{ARUJÁ-SP}
\data{2024}
\instituicao{%
  FLAM - FACULDADE LATINO AMERICANA
}
\tipotrabalho{QUESTÃO ABERTA 03}
% O preambulo deve conter o tipo do trabalho, o objetivo, 
% o nome da instituição e a área de concentração 
\preambulo{Trabalho da disciplina de Metodologia da Pesquisa 1, solicitado pela profa. Dra. Inês Murad.}
% ---


% ---
% Configurações de aparência do PDF final

% alterando o aspecto da cor azul
\definecolor{blue}{RGB}{41,5,195}

% informações do PDF
\makeatletter
\hypersetup{
     	%pagebackref=true,
		pdftitle={\@title}, 
		pdfauthor={\@author},
    	pdfsubject={\imprimirpreambulo},
	    pdfcreator={LaTeX with abnTeX2},
		pdfkeywords={trabalho acadêmico},
		colorlinks=true,       		% false: boxed links; true: colored links
    	linkcolor=blue,          	% color of internal links
    	citecolor=blue,        		% color of links to bibliography
    	filecolor=magenta,      		% color of file links
		urlcolor=blue,
		bookmarksdepth=4
}
\makeatother
% ---
% ---
% compila o indice
% ---
\makeindex
% ---

% ----
% Início do documento
% ----
\begin{document}

\citeoption{abnt-full-initials=yes}


% Seleciona o idioma do documento (conforme pacotes do babel)
%\selectlanguage{english}
\selectlanguage{brazil}
% ----------------------------------------------------------
% ELEMENTOS PRÉ-TEXTUAIS
% ----------------------------------------------------------
% \pretextual

% ---
% Capa
% ---
\imprimircapa
% ---

% ---
% Folha de rosto
% (o * indica que haverá a ficha bibliográfica)
% ---
% \imprimirfolhaderosto*
\imprimirfolhaderosto
% ---

% ---
% inserir o sumario
---
% \pdfbookmark[0]{\contentsname}{toc}
% \tableofcontents*
% \cleardoublepage
% ---

% ----------------------------------------------------------
% ELEMENTOS TEXTUAIS
% ----------------------------------------------------------
\textual
\pagestyle{simple}

% ----------------------------------------------------------
% Introdução (mas presente no Sumário)

\section*{TEMA}
A formação de comunidades cristãs sob uma perspectiva escatológica: consequências da expectativa escatológica nas origens de uma igreja.

\section*{JUSTIFICATIVA}
Atualmente, igrejas e comunidades cristãs costumam tratar da vida em comunidade de forma secularizada, onde existe uma expectativa de perenidade quanto aos bens materiais e estruturas sociais. Estas atitudes são, em parte, diferentes de como comunidades cristãs que nascem a partir de uma expectativa escatológica, ou seja, uma expectativa do fim de todas as coisas, tais como a forma de gestão de recursos entre seus membros, a priorização da pregação do Evangelho e da segunda volta de Cristo, entre outros. Entendendo como essas igrejas e comunidades foram formadas e também compreendendo como suas noções da brevidade do fim permearam a formação de suas estruturas eclesiásticas, poderemos extrair informações e compreensões aplicáveis a comunidades cristãs que, hoje, não possuem a mesma perspectiva.

\section*{PROBLEMA}
Como o entendimento e a fé que a segunda volta de Jesus Cristo e o fim de todas as coisas influenciou a criação de comunidades e igrejas cristãs ao longo da história.

\section*{OBJETIVO GERAL}
Conhecer as estruturas eclesiásticas e o modus operandi dos membros de comunidades cristãs que se formaram sob um contexto de extrema influência escatológica pela história da Igreja.

\section*{OBJETIVO ESPECÍFICOS}
\begin{itemize}
    \item Apontar comunidades cristãs e igrejas específicas que se formaram sob este contexto, pela história.
    \item Identificar características eclesiásticas e sociais entre essas comunidades e igrejas.
    \item Demonstrar as diferenças eclesiásticas e sociais entre igrejas contemporâneas e estas igrejas.
\end{itemize}

\section*{METODOLOGIA}
Para elaborar o trabalho, será utilizado os método de pesquisa documental e descritiva que buscará detalhar e montar qual é o cenário propício para a formação de uma comunidade onde seus indivíduos, em conjunto, aguardam a volta do Messias baseado no estudo de comunidades cristãs que tiveram esta mesma crença como principal. O empenho estará em entender não só as circunstâncias prévias daqueles indivíduos que se juntaram em uma igreja ou comunidade mas também em como essa comunidade existia em si mesma e dentro da sociedade que estavam inseridos.

\section*{SUMÁRIO PROVISÓRIO}
\begin{itemize}
    \item Introdução: Resumo do projeto e resumo da conclusão.
    \item Capítulo 1: Comunidades cristãs e igrejas formadas pela expectativa da segunda volta de Cristo.
    \item Capítulo 2: Características eclesiásticas e sociais compartilhadas.
    \item Capítulo 3: Igrejas contemporâneas em contraposição a expectativa da segunda volta de Cristo.
    \item Conclusão: Retomada do tema e do problema, demonstrando como o objetivo geral foi alcançado.
\end{itemize}

\section*{REFERENCIAL TEÓRICO INICIAL}
\nocite{CAIRNS}
\nocite{DOOL}
\nocite{WEINLICK}
\nocite{ATWOOD}
\nocite{LOPES}

% ----------------------------------------------------------

\renewcommand{\bibname}{{}}
\bibliography{template_flam.bib}

\end{document}
